\documentclass[11pt]{article}

    \usepackage[breakable]{tcolorbox}
    \usepackage{parskip} % Stop auto-indenting (to mimic markdown behaviour)
    \usepackage{ctex}
    \usepackage{iftex}
    \ifPDFTeX
    	\usepackage[T1]{fontenc}
    	\usepackage{mathpazo}
    \else
    	\usepackage{fontspec}
    \fi

    % Basic figure setup, for now with no caption control since it's done
    % automatically by Pandoc (which extracts ![](path) syntax from Markdown).
    \usepackage{graphicx}
    % Maintain compatibility with old templates. Remove in nbconvert 6.0
    \let\Oldincludegraphics\includegraphics
    % Ensure that by default, figures have no caption (until we provide a
    % proper Figure object with a Caption API and a way to capture that
    % in the conversion process - todo).
    \usepackage{caption}
    \DeclareCaptionFormat{nocaption}{}
    \captionsetup{format=nocaption,aboveskip=0pt,belowskip=0pt}

    \usepackage{float}
    \floatplacement{figure}{H} % forces figures to be placed at the correct location
    \usepackage{xcolor} % Allow colors to be defined
    \usepackage{enumerate} % Needed for markdown enumerations to work
    \usepackage{geometry} % Used to adjust the document margins
    \usepackage{amsmath} % Equations
    \usepackage{amssymb} % Equations
    \usepackage{textcomp} % defines textquotesingle
    % Hack from http://tex.stackexchange.com/a/47451/13684:
    \AtBeginDocument{%
        \def\PYZsq{\textquotesingle}% Upright quotes in Pygmentized code
    }
    \usepackage{upquote} % Upright quotes for verbatim code
    \usepackage{eurosym} % defines \euro
    \usepackage[mathletters]{ucs} % Extended unicode (utf-8) support
    \usepackage{fancyvrb} % verbatim replacement that allows latex
    \usepackage{grffile} % extends the file name processing of package graphics 
                         % to support a larger range
    \makeatletter % fix for old versions of grffile with XeLaTeX
    \@ifpackagelater{grffile}{2019/11/01}
    {
      % Do nothing on new versions
    }
    {
      \def\Gread@@xetex#1{%
        \IfFileExists{"\Gin@base".bb}%
        {\Gread@eps{\Gin@base.bb}}%
        {\Gread@@xetex@aux#1}%
      }
    }
    \makeatother
    \usepackage[Export]{adjustbox} % Used to constrain images to a maximum size
    \adjustboxset{max size={0.9\linewidth}{0.9\paperheight}}

    % The hyperref package gives us a pdf with properly built
    % internal navigation ('pdf bookmarks' for the table of contents,
    % internal cross-reference links, web links for URLs, etc.)
    \usepackage{hyperref}
    % The default LaTeX title has an obnoxious amount of whitespace. By default,
    % titling removes some of it. It also provides customization options.
    \usepackage{titling}
    \usepackage{longtable} % longtable support required by pandoc >1.10
    \usepackage{booktabs}  % table support for pandoc > 1.12.2
    \usepackage[inline]{enumitem} % IRkernel/repr support (it uses the enumerate* environment)
    \usepackage[normalem]{ulem} % ulem is needed to support strikethroughs (\sout)
                                % normalem makes italics be italics, not underlines
    \usepackage{mathrsfs}
    

    
    % Colors for the hyperref package
    \definecolor{urlcolor}{rgb}{0,.145,.698}
    \definecolor{linkcolor}{rgb}{.71,0.21,0.01}
    \definecolor{citecolor}{rgb}{.12,.54,.11}

    % ANSI colors
    \definecolor{ansi-black}{HTML}{3E424D}
    \definecolor{ansi-black-intense}{HTML}{282C36}
    \definecolor{ansi-red}{HTML}{E75C58}
    \definecolor{ansi-red-intense}{HTML}{B22B31}
    \definecolor{ansi-green}{HTML}{00A250}
    \definecolor{ansi-green-intense}{HTML}{007427}
    \definecolor{ansi-yellow}{HTML}{DDB62B}
    \definecolor{ansi-yellow-intense}{HTML}{B27D12}
    \definecolor{ansi-blue}{HTML}{208FFB}
    \definecolor{ansi-blue-intense}{HTML}{0065CA}
    \definecolor{ansi-magenta}{HTML}{D160C4}
    \definecolor{ansi-magenta-intense}{HTML}{A03196}
    \definecolor{ansi-cyan}{HTML}{60C6C8}
    \definecolor{ansi-cyan-intense}{HTML}{258F8F}
    \definecolor{ansi-white}{HTML}{C5C1B4}
    \definecolor{ansi-white-intense}{HTML}{A1A6B2}
    \definecolor{ansi-default-inverse-fg}{HTML}{FFFFFF}
    \definecolor{ansi-default-inverse-bg}{HTML}{000000}

    % common color for the border for error outputs.
    \definecolor{outerrorbackground}{HTML}{FFDFDF}

    % commands and environments needed by pandoc snippets
    % extracted from the output of `pandoc -s`
    \providecommand{\tightlist}{%
      \setlength{\itemsep}{0pt}\setlength{\parskip}{0pt}}
    \DefineVerbatimEnvironment{Highlighting}{Verbatim}{commandchars=\\\{\}}
    % Add ',fontsize=\small' for more characters per line
    \newenvironment{Shaded}{}{}
    \newcommand{\KeywordTok}[1]{\textcolor[rgb]{0.00,0.44,0.13}{\textbf{{#1}}}}
    \newcommand{\DataTypeTok}[1]{\textcolor[rgb]{0.56,0.13,0.00}{{#1}}}
    \newcommand{\DecValTok}[1]{\textcolor[rgb]{0.25,0.63,0.44}{{#1}}}
    \newcommand{\BaseNTok}[1]{\textcolor[rgb]{0.25,0.63,0.44}{{#1}}}
    \newcommand{\FloatTok}[1]{\textcolor[rgb]{0.25,0.63,0.44}{{#1}}}
    \newcommand{\CharTok}[1]{\textcolor[rgb]{0.25,0.44,0.63}{{#1}}}
    \newcommand{\StringTok}[1]{\textcolor[rgb]{0.25,0.44,0.63}{{#1}}}
    \newcommand{\CommentTok}[1]{\textcolor[rgb]{0.38,0.63,0.69}{\textit{{#1}}}}
    \newcommand{\OtherTok}[1]{\textcolor[rgb]{0.00,0.44,0.13}{{#1}}}
    \newcommand{\AlertTok}[1]{\textcolor[rgb]{1.00,0.00,0.00}{\textbf{{#1}}}}
    \newcommand{\FunctionTok}[1]{\textcolor[rgb]{0.02,0.16,0.49}{{#1}}}
    \newcommand{\RegionMarkerTok}[1]{{#1}}
    \newcommand{\ErrorTok}[1]{\textcolor[rgb]{1.00,0.00,0.00}{\textbf{{#1}}}}
    \newcommand{\NormalTok}[1]{{#1}}
    
    % Additional commands for more recent versions of Pandoc
    \newcommand{\ConstantTok}[1]{\textcolor[rgb]{0.53,0.00,0.00}{{#1}}}
    \newcommand{\SpecialCharTok}[1]{\textcolor[rgb]{0.25,0.44,0.63}{{#1}}}
    \newcommand{\VerbatimStringTok}[1]{\textcolor[rgb]{0.25,0.44,0.63}{{#1}}}
    \newcommand{\SpecialStringTok}[1]{\textcolor[rgb]{0.73,0.40,0.53}{{#1}}}
    \newcommand{\ImportTok}[1]{{#1}}
    \newcommand{\DocumentationTok}[1]{\textcolor[rgb]{0.73,0.13,0.13}{\textit{{#1}}}}
    \newcommand{\AnnotationTok}[1]{\textcolor[rgb]{0.38,0.63,0.69}{\textbf{\textit{{#1}}}}}
    \newcommand{\CommentVarTok}[1]{\textcolor[rgb]{0.38,0.63,0.69}{\textbf{\textit{{#1}}}}}
    \newcommand{\VariableTok}[1]{\textcolor[rgb]{0.10,0.09,0.49}{{#1}}}
    \newcommand{\ControlFlowTok}[1]{\textcolor[rgb]{0.00,0.44,0.13}{\textbf{{#1}}}}
    \newcommand{\OperatorTok}[1]{\textcolor[rgb]{0.40,0.40,0.40}{{#1}}}
    \newcommand{\BuiltInTok}[1]{{#1}}
    \newcommand{\ExtensionTok}[1]{{#1}}
    \newcommand{\PreprocessorTok}[1]{\textcolor[rgb]{0.74,0.48,0.00}{{#1}}}
    \newcommand{\AttributeTok}[1]{\textcolor[rgb]{0.49,0.56,0.16}{{#1}}}
    \newcommand{\InformationTok}[1]{\textcolor[rgb]{0.38,0.63,0.69}{\textbf{\textit{{#1}}}}}
    \newcommand{\WarningTok}[1]{\textcolor[rgb]{0.38,0.63,0.69}{\textbf{\textit{{#1}}}}}
    
    
    % Define a nice break command that doesn't care if a line doesn't already
    % exist.
    \def\br{\hspace*{\fill} \\* }
    % Math Jax compatibility definitions
    \def\gt{>}
    \def\lt{<}
    \let\Oldtex\TeX
    \let\Oldlatex\LaTeX
    \renewcommand{\TeX}{\textrm{\Oldtex}}
    \renewcommand{\LaTeX}{\textrm{\Oldlatex}}
    % Document parameters
    % Document title
    \title{lstm\_apply}
    
    
    
    
    
% Pygments definitions
\makeatletter
\def\PY@reset{\let\PY@it=\relax \let\PY@bf=\relax%
    \let\PY@ul=\relax \let\PY@tc=\relax%
    \let\PY@bc=\relax \let\PY@ff=\relax}
\def\PY@tok#1{\csname PY@tok@#1\endcsname}
\def\PY@toks#1+{\ifx\relax#1\empty\else%
    \PY@tok{#1}\expandafter\PY@toks\fi}
\def\PY@do#1{\PY@bc{\PY@tc{\PY@ul{%
    \PY@it{\PY@bf{\PY@ff{#1}}}}}}}
\def\PY#1#2{\PY@reset\PY@toks#1+\relax+\PY@do{#2}}

\@namedef{PY@tok@w}{\def\PY@tc##1{\textcolor[rgb]{0.73,0.73,0.73}{##1}}}
\@namedef{PY@tok@c}{\let\PY@it=\textit\def\PY@tc##1{\textcolor[rgb]{0.24,0.48,0.48}{##1}}}
\@namedef{PY@tok@cp}{\def\PY@tc##1{\textcolor[rgb]{0.61,0.40,0.00}{##1}}}
\@namedef{PY@tok@k}{\let\PY@bf=\textbf\def\PY@tc##1{\textcolor[rgb]{0.00,0.50,0.00}{##1}}}
\@namedef{PY@tok@kp}{\def\PY@tc##1{\textcolor[rgb]{0.00,0.50,0.00}{##1}}}
\@namedef{PY@tok@kt}{\def\PY@tc##1{\textcolor[rgb]{0.69,0.00,0.25}{##1}}}
\@namedef{PY@tok@o}{\def\PY@tc##1{\textcolor[rgb]{0.40,0.40,0.40}{##1}}}
\@namedef{PY@tok@ow}{\let\PY@bf=\textbf\def\PY@tc##1{\textcolor[rgb]{0.67,0.13,1.00}{##1}}}
\@namedef{PY@tok@nb}{\def\PY@tc##1{\textcolor[rgb]{0.00,0.50,0.00}{##1}}}
\@namedef{PY@tok@nf}{\def\PY@tc##1{\textcolor[rgb]{0.00,0.00,1.00}{##1}}}
\@namedef{PY@tok@nc}{\let\PY@bf=\textbf\def\PY@tc##1{\textcolor[rgb]{0.00,0.00,1.00}{##1}}}
\@namedef{PY@tok@nn}{\let\PY@bf=\textbf\def\PY@tc##1{\textcolor[rgb]{0.00,0.00,1.00}{##1}}}
\@namedef{PY@tok@ne}{\let\PY@bf=\textbf\def\PY@tc##1{\textcolor[rgb]{0.80,0.25,0.22}{##1}}}
\@namedef{PY@tok@nv}{\def\PY@tc##1{\textcolor[rgb]{0.10,0.09,0.49}{##1}}}
\@namedef{PY@tok@no}{\def\PY@tc##1{\textcolor[rgb]{0.53,0.00,0.00}{##1}}}
\@namedef{PY@tok@nl}{\def\PY@tc##1{\textcolor[rgb]{0.46,0.46,0.00}{##1}}}
\@namedef{PY@tok@ni}{\let\PY@bf=\textbf\def\PY@tc##1{\textcolor[rgb]{0.44,0.44,0.44}{##1}}}
\@namedef{PY@tok@na}{\def\PY@tc##1{\textcolor[rgb]{0.41,0.47,0.13}{##1}}}
\@namedef{PY@tok@nt}{\let\PY@bf=\textbf\def\PY@tc##1{\textcolor[rgb]{0.00,0.50,0.00}{##1}}}
\@namedef{PY@tok@nd}{\def\PY@tc##1{\textcolor[rgb]{0.67,0.13,1.00}{##1}}}
\@namedef{PY@tok@s}{\def\PY@tc##1{\textcolor[rgb]{0.73,0.13,0.13}{##1}}}
\@namedef{PY@tok@sd}{\let\PY@it=\textit\def\PY@tc##1{\textcolor[rgb]{0.73,0.13,0.13}{##1}}}
\@namedef{PY@tok@si}{\let\PY@bf=\textbf\def\PY@tc##1{\textcolor[rgb]{0.64,0.35,0.47}{##1}}}
\@namedef{PY@tok@se}{\let\PY@bf=\textbf\def\PY@tc##1{\textcolor[rgb]{0.67,0.36,0.12}{##1}}}
\@namedef{PY@tok@sr}{\def\PY@tc##1{\textcolor[rgb]{0.64,0.35,0.47}{##1}}}
\@namedef{PY@tok@ss}{\def\PY@tc##1{\textcolor[rgb]{0.10,0.09,0.49}{##1}}}
\@namedef{PY@tok@sx}{\def\PY@tc##1{\textcolor[rgb]{0.00,0.50,0.00}{##1}}}
\@namedef{PY@tok@m}{\def\PY@tc##1{\textcolor[rgb]{0.40,0.40,0.40}{##1}}}
\@namedef{PY@tok@gh}{\let\PY@bf=\textbf\def\PY@tc##1{\textcolor[rgb]{0.00,0.00,0.50}{##1}}}
\@namedef{PY@tok@gu}{\let\PY@bf=\textbf\def\PY@tc##1{\textcolor[rgb]{0.50,0.00,0.50}{##1}}}
\@namedef{PY@tok@gd}{\def\PY@tc##1{\textcolor[rgb]{0.63,0.00,0.00}{##1}}}
\@namedef{PY@tok@gi}{\def\PY@tc##1{\textcolor[rgb]{0.00,0.52,0.00}{##1}}}
\@namedef{PY@tok@gr}{\def\PY@tc##1{\textcolor[rgb]{0.89,0.00,0.00}{##1}}}
\@namedef{PY@tok@ge}{\let\PY@it=\textit}
\@namedef{PY@tok@gs}{\let\PY@bf=\textbf}
\@namedef{PY@tok@gp}{\let\PY@bf=\textbf\def\PY@tc##1{\textcolor[rgb]{0.00,0.00,0.50}{##1}}}
\@namedef{PY@tok@go}{\def\PY@tc##1{\textcolor[rgb]{0.44,0.44,0.44}{##1}}}
\@namedef{PY@tok@gt}{\def\PY@tc##1{\textcolor[rgb]{0.00,0.27,0.87}{##1}}}
\@namedef{PY@tok@err}{\def\PY@bc##1{{\setlength{\fboxsep}{\string -\fboxrule}\fcolorbox[rgb]{1.00,0.00,0.00}{1,1,1}{\strut ##1}}}}
\@namedef{PY@tok@kc}{\let\PY@bf=\textbf\def\PY@tc##1{\textcolor[rgb]{0.00,0.50,0.00}{##1}}}
\@namedef{PY@tok@kd}{\let\PY@bf=\textbf\def\PY@tc##1{\textcolor[rgb]{0.00,0.50,0.00}{##1}}}
\@namedef{PY@tok@kn}{\let\PY@bf=\textbf\def\PY@tc##1{\textcolor[rgb]{0.00,0.50,0.00}{##1}}}
\@namedef{PY@tok@kr}{\let\PY@bf=\textbf\def\PY@tc##1{\textcolor[rgb]{0.00,0.50,0.00}{##1}}}
\@namedef{PY@tok@bp}{\def\PY@tc##1{\textcolor[rgb]{0.00,0.50,0.00}{##1}}}
\@namedef{PY@tok@fm}{\def\PY@tc##1{\textcolor[rgb]{0.00,0.00,1.00}{##1}}}
\@namedef{PY@tok@vc}{\def\PY@tc##1{\textcolor[rgb]{0.10,0.09,0.49}{##1}}}
\@namedef{PY@tok@vg}{\def\PY@tc##1{\textcolor[rgb]{0.10,0.09,0.49}{##1}}}
\@namedef{PY@tok@vi}{\def\PY@tc##1{\textcolor[rgb]{0.10,0.09,0.49}{##1}}}
\@namedef{PY@tok@vm}{\def\PY@tc##1{\textcolor[rgb]{0.10,0.09,0.49}{##1}}}
\@namedef{PY@tok@sa}{\def\PY@tc##1{\textcolor[rgb]{0.73,0.13,0.13}{##1}}}
\@namedef{PY@tok@sb}{\def\PY@tc##1{\textcolor[rgb]{0.73,0.13,0.13}{##1}}}
\@namedef{PY@tok@sc}{\def\PY@tc##1{\textcolor[rgb]{0.73,0.13,0.13}{##1}}}
\@namedef{PY@tok@dl}{\def\PY@tc##1{\textcolor[rgb]{0.73,0.13,0.13}{##1}}}
\@namedef{PY@tok@s2}{\def\PY@tc##1{\textcolor[rgb]{0.73,0.13,0.13}{##1}}}
\@namedef{PY@tok@sh}{\def\PY@tc##1{\textcolor[rgb]{0.73,0.13,0.13}{##1}}}
\@namedef{PY@tok@s1}{\def\PY@tc##1{\textcolor[rgb]{0.73,0.13,0.13}{##1}}}
\@namedef{PY@tok@mb}{\def\PY@tc##1{\textcolor[rgb]{0.40,0.40,0.40}{##1}}}
\@namedef{PY@tok@mf}{\def\PY@tc##1{\textcolor[rgb]{0.40,0.40,0.40}{##1}}}
\@namedef{PY@tok@mh}{\def\PY@tc##1{\textcolor[rgb]{0.40,0.40,0.40}{##1}}}
\@namedef{PY@tok@mi}{\def\PY@tc##1{\textcolor[rgb]{0.40,0.40,0.40}{##1}}}
\@namedef{PY@tok@il}{\def\PY@tc##1{\textcolor[rgb]{0.40,0.40,0.40}{##1}}}
\@namedef{PY@tok@mo}{\def\PY@tc##1{\textcolor[rgb]{0.40,0.40,0.40}{##1}}}
\@namedef{PY@tok@ch}{\let\PY@it=\textit\def\PY@tc##1{\textcolor[rgb]{0.24,0.48,0.48}{##1}}}
\@namedef{PY@tok@cm}{\let\PY@it=\textit\def\PY@tc##1{\textcolor[rgb]{0.24,0.48,0.48}{##1}}}
\@namedef{PY@tok@cpf}{\let\PY@it=\textit\def\PY@tc##1{\textcolor[rgb]{0.24,0.48,0.48}{##1}}}
\@namedef{PY@tok@c1}{\let\PY@it=\textit\def\PY@tc##1{\textcolor[rgb]{0.24,0.48,0.48}{##1}}}
\@namedef{PY@tok@cs}{\let\PY@it=\textit\def\PY@tc##1{\textcolor[rgb]{0.24,0.48,0.48}{##1}}}

\def\PYZbs{\char`\\}
\def\PYZus{\char`\_}
\def\PYZob{\char`\{}
\def\PYZcb{\char`\}}
\def\PYZca{\char`\^}
\def\PYZam{\char`\&}
\def\PYZlt{\char`\<}
\def\PYZgt{\char`\>}
\def\PYZsh{\char`\#}
\def\PYZpc{\char`\%}
\def\PYZdl{\char`\$}
\def\PYZhy{\char`\-}
\def\PYZsq{\char`\'}
\def\PYZdq{\char`\"}
\def\PYZti{\char`\~}
% for compatibility with earlier versions
\def\PYZat{@}
\def\PYZlb{[}
\def\PYZrb{]}
\makeatother


    % For linebreaks inside Verbatim environment from package fancyvrb. 
    \makeatletter
        \newbox\Wrappedcontinuationbox 
        \newbox\Wrappedvisiblespacebox 
        \newcommand*\Wrappedvisiblespace {\textcolor{red}{\textvisiblespace}} 
        \newcommand*\Wrappedcontinuationsymbol {\textcolor{red}{\llap{\tiny$\m@th\hookrightarrow$}}} 
        \newcommand*\Wrappedcontinuationindent {3ex } 
        \newcommand*\Wrappedafterbreak {\kern\Wrappedcontinuationindent\copy\Wrappedcontinuationbox} 
        % Take advantage of the already applied Pygments mark-up to insert 
        % potential linebreaks for TeX processing. 
        %        {, <, #, %, $, ' and ": go to next line. 
        %        _, }, ^, &, >, - and ~: stay at end of broken line. 
        % Use of \textquotesingle for straight quote. 
        \newcommand*\Wrappedbreaksatspecials {% 
            \def\PYGZus{\discretionary{\char`\_}{\Wrappedafterbreak}{\char`\_}}% 
            \def\PYGZob{\discretionary{}{\Wrappedafterbreak\char`\{}{\char`\{}}% 
            \def\PYGZcb{\discretionary{\char`\}}{\Wrappedafterbreak}{\char`\}}}% 
            \def\PYGZca{\discretionary{\char`\^}{\Wrappedafterbreak}{\char`\^}}% 
            \def\PYGZam{\discretionary{\char`\&}{\Wrappedafterbreak}{\char`\&}}% 
            \def\PYGZlt{\discretionary{}{\Wrappedafterbreak\char`\<}{\char`\<}}% 
            \def\PYGZgt{\discretionary{\char`\>}{\Wrappedafterbreak}{\char`\>}}% 
            \def\PYGZsh{\discretionary{}{\Wrappedafterbreak\char`\#}{\char`\#}}% 
            \def\PYGZpc{\discretionary{}{\Wrappedafterbreak\char`\%}{\char`\%}}% 
            \def\PYGZdl{\discretionary{}{\Wrappedafterbreak\char`\$}{\char`\$}}% 
            \def\PYGZhy{\discretionary{\char`\-}{\Wrappedafterbreak}{\char`\-}}% 
            \def\PYGZsq{\discretionary{}{\Wrappedafterbreak\textquotesingle}{\textquotesingle}}% 
            \def\PYGZdq{\discretionary{}{\Wrappedafterbreak\char`\"}{\char`\"}}% 
            \def\PYGZti{\discretionary{\char`\~}{\Wrappedafterbreak}{\char`\~}}% 
        } 
        % Some characters . , ; ? ! / are not pygmentized. 
        % This macro makes them "active" and they will insert potential linebreaks 
        \newcommand*\Wrappedbreaksatpunct {% 
            \lccode`\~`\.\lowercase{\def~}{\discretionary{\hbox{\char`\.}}{\Wrappedafterbreak}{\hbox{\char`\.}}}% 
            \lccode`\~`\,\lowercase{\def~}{\discretionary{\hbox{\char`\,}}{\Wrappedafterbreak}{\hbox{\char`\,}}}% 
            \lccode`\~`\;\lowercase{\def~}{\discretionary{\hbox{\char`\;}}{\Wrappedafterbreak}{\hbox{\char`\;}}}% 
            \lccode`\~`\:\lowercase{\def~}{\discretionary{\hbox{\char`\:}}{\Wrappedafterbreak}{\hbox{\char`\:}}}% 
            \lccode`\~`\?\lowercase{\def~}{\discretionary{\hbox{\char`\?}}{\Wrappedafterbreak}{\hbox{\char`\?}}}% 
            \lccode`\~`\!\lowercase{\def~}{\discretionary{\hbox{\char`\!}}{\Wrappedafterbreak}{\hbox{\char`\!}}}% 
            \lccode`\~`\/\lowercase{\def~}{\discretionary{\hbox{\char`\/}}{\Wrappedafterbreak}{\hbox{\char`\/}}}% 
            \catcode`\.\active
            \catcode`\,\active 
            \catcode`\;\active
            \catcode`\:\active
            \catcode`\?\active
            \catcode`\!\active
            \catcode`\/\active 
            \lccode`\~`\~ 	
        }
    \makeatother

    \let\OriginalVerbatim=\Verbatim
    \makeatletter
    \renewcommand{\Verbatim}[1][1]{%
        %\parskip\z@skip
        \sbox\Wrappedcontinuationbox {\Wrappedcontinuationsymbol}%
        \sbox\Wrappedvisiblespacebox {\FV@SetupFont\Wrappedvisiblespace}%
        \def\FancyVerbFormatLine ##1{\hsize\linewidth
            \vtop{\raggedright\hyphenpenalty\z@\exhyphenpenalty\z@
                \doublehyphendemerits\z@\finalhyphendemerits\z@
                \strut ##1\strut}%
        }%
        % If the linebreak is at a space, the latter will be displayed as visible
        % space at end of first line, and a continuation symbol starts next line.
        % Stretch/shrink are however usually zero for typewriter font.
        \def\FV@Space {%
            \nobreak\hskip\z@ plus\fontdimen3\font minus\fontdimen4\font
            \discretionary{\copy\Wrappedvisiblespacebox}{\Wrappedafterbreak}
            {\kern\fontdimen2\font}%
        }%
        
        % Allow breaks at special characters using \PYG... macros.
        \Wrappedbreaksatspecials
        % Breaks at punctuation characters . , ; ? ! and / need catcode=\active 	
        \OriginalVerbatim[#1,codes*=\Wrappedbreaksatpunct]%
    }
    \makeatother

    % Exact colors from NB
    \definecolor{incolor}{HTML}{303F9F}
    \definecolor{outcolor}{HTML}{D84315}
    \definecolor{cellborder}{HTML}{CFCFCF}
    \definecolor{cellbackground}{HTML}{F7F7F7}
    
    % prompt
    \makeatletter
    \newcommand{\boxspacing}{\kern\kvtcb@left@rule\kern\kvtcb@boxsep}
    \makeatother
    \newcommand{\prompt}[4]{
        {\ttfamily\llap{{\color{#2}[#3]:\hspace{3pt}#4}}\vspace{-\baselineskip}}
    }
    

    
    % Prevent overflowing lines due to hard-to-break entities
    \sloppy 
    % Setup hyperref package
    \hypersetup{
      breaklinks=true,  % so long urls are correctly broken across lines
      colorlinks=true,
      urlcolor=urlcolor,
      linkcolor=linkcolor,
      citecolor=citecolor,
      }
    % Slightly bigger margins than the latex defaults
    
    \geometry{verbose,tmargin=1in,bmargin=1in,lmargin=1in,rmargin=1in}
    
    

\begin{document}
    
    \maketitle
    
    

    
    \hypertarget{lstmux7684demo}{%
\section{lstm的demo}\label{lstmux7684demo}}

    \begin{tcolorbox}[breakable, size=fbox, boxrule=1pt, pad at break*=1mm,colback=cellbackground, colframe=cellborder]
\prompt{In}{incolor}{1}{\boxspacing}
\begin{Verbatim}[commandchars=\\\{\}]
\PY{k+kn}{import} \PY{n+nn}{numpy} \PY{k}{as} \PY{n+nn}{np}
\PY{k+kn}{import} \PY{n+nn}{torch}
\PY{k+kn}{from} \PY{n+nn}{torch} \PY{k+kn}{import} \PY{n}{nn}
\PY{k+kn}{import} \PY{n+nn}{matplotlib}\PY{n+nn}{.}\PY{n+nn}{pyplot} \PY{k}{as} \PY{n+nn}{plt}
\end{Verbatim}
\end{tcolorbox}

    \hypertarget{define-the-network}{%
\subsection{define the network}\label{define-the-network}}

    \hypertarget{ux539fux7406}{%
\subsubsection{原理}\label{ux539fux7406}}

\begin{center}
	\adjustimage{max size={0.9\linewidth}{0.5\paperheight}}{lstm.jpg}
\end{center}


\[
\begin{array}{ll} \\
    i_t = \sigma(W_{ii} x_t + b_{ii} + W_{hi} h_{t-1} + b_{hi}) \\
    f_t = \sigma(W_{if} x_t + b_{if} + W_{hf} h_{t-1} + b_{hf}) \\
    g_t = \tanh(W_{ig} x_t + b_{ig} + W_{hg} h_{t-1} + b_{hg}) \\
    o_t = \sigma(W_{io} x_t + b_{io} + W_{ho} h_{t-1} + b_{ho}) \\
    c_t = f_t \odot c_{t-1} + i_t \odot g_t \\
    h_t = o_t \odot \tanh(c_t) \\
\end{array}
\]

\begin{quote}
注:上面的下标是指从什么到什么,比如\(W_{ii}\)表示从输入到输入的权重,\(W_{ig}\)表示从输入到门的权重,\(W_{hg}\)表示从隐藏层到门的权重,\(W_{ho}\)表示从隐藏层到输出的权重,\(W_{hf}\)表示从隐藏层到遗忘门的权重,\(W_{hi}\)表示从隐藏层到输入门的权重。\(b_{ii}\)表示从输入到输入的偏置,\(b_{ig}\)表示从输入到门的偏置,\(b_{hg}\)表示从隐藏层到门的偏置,\(b_{ho}\)表示从隐藏层到输出的偏置,\(b_{hf}\)表示从隐藏层到遗忘门的偏置,\(b_{hi}\)表示从隐藏层到输入门的偏置。\(i_t\)表示输入门,\(f_t\)表示遗忘门,\(g_t\)表示门,\(o_t\)表示输出门,\(c_t\)表示记忆单元,\(h_t\)表示隐藏层。
\end{quote}

    \begin{tcolorbox}[breakable, size=fbox, boxrule=1pt, pad at break*=1mm,colback=cellbackground, colframe=cellborder]
\prompt{In}{incolor}{2}{\boxspacing}
\begin{Verbatim}[commandchars=\\\{\}]
\PY{k}{class} \PY{n+nc}{RegLSTM}\PY{p}{(}\PY{n}{nn}\PY{o}{.}\PY{n}{Module}\PY{p}{)}\PY{p}{:}
    \PY{k}{def} \PY{n+nf+fm}{\PYZus{}\PYZus{}init\PYZus{}\PYZus{}}\PY{p}{(}\PY{n+nb+bp}{self}\PY{p}{,} \PY{n}{inp\PYZus{}dim}\PY{p}{,} \PY{n}{out\PYZus{}dim}\PY{p}{,} \PY{n}{mid\PYZus{}dim}\PY{p}{,} \PY{n}{mid\PYZus{}layers}\PY{p}{)}\PY{p}{:}
        \PY{n+nb}{super}\PY{p}{(}\PY{n}{RegLSTM}\PY{p}{,} \PY{n+nb+bp}{self}\PY{p}{)}\PY{o}{.}\PY{n+nf+fm}{\PYZus{}\PYZus{}init\PYZus{}\PYZus{}}\PY{p}{(}\PY{p}{)}
        \PY{n+nb+bp}{self}\PY{o}{.}\PY{n}{rnn} \PY{o}{=} \PY{n}{nn}\PY{o}{.}\PY{n}{LSTM}\PY{p}{(}\PY{n}{inp\PYZus{}dim}\PY{p}{,} \PY{n}{mid\PYZus{}dim}\PY{p}{,} \PY{n}{mid\PYZus{}layers}\PY{p}{)}  
        \PY{c+c1}{\PYZsh{} rnn layer 在自然语言处理中,第一个参数通常是embedding的维度,第二个参数是隐藏层的维度,第三个参数是层数}
        \PY{n+nb+bp}{self}\PY{o}{.}\PY{n}{reg} \PY{o}{=} \PY{n}{nn}\PY{o}{.}\PY{n}{Sequential}\PY{p}{(}
            \PY{n}{nn}\PY{o}{.}\PY{n}{Linear}\PY{p}{(}\PY{n}{mid\PYZus{}dim}\PY{p}{,} \PY{n}{mid\PYZus{}dim}\PY{p}{)}\PY{p}{,}
            \PY{n}{nn}\PY{o}{.}\PY{n}{Tanh}\PY{p}{(}\PY{p}{)}\PY{p}{,}
            \PY{n}{nn}\PY{o}{.}\PY{n}{Linear}\PY{p}{(}\PY{n}{mid\PYZus{}dim}\PY{p}{,} \PY{n}{out\PYZus{}dim}\PY{p}{)}\PY{p}{,}
        \PY{p}{)}  \PY{c+c1}{\PYZsh{} regression}

    \PY{k}{def} \PY{n+nf}{forward}\PY{p}{(}\PY{n+nb+bp}{self}\PY{p}{,} \PY{n}{x}\PY{p}{)}\PY{p}{:}
        \PY{n}{y} \PY{o}{=} \PY{n+nb+bp}{self}\PY{o}{.}\PY{n}{rnn}\PY{p}{(}\PY{n}{x}\PY{p}{)}\PY{p}{[}\PY{l+m+mi}{0}\PY{p}{]}  \PY{c+c1}{\PYZsh{} y, (h, c) = self.rnn(x)}

        \PY{n}{seq\PYZus{}len}\PY{p}{,} \PY{n}{batch\PYZus{}size}\PY{p}{,} \PY{n}{hid\PYZus{}dim} \PY{o}{=} \PY{n}{y}\PY{o}{.}\PY{n}{shape}
        \PY{n}{y} \PY{o}{=} \PY{n}{y}\PY{o}{.}\PY{n}{view}\PY{p}{(}\PY{o}{\PYZhy{}}\PY{l+m+mi}{1}\PY{p}{,} \PY{n}{hid\PYZus{}dim}\PY{p}{)} \PY{c+c1}{\PYZsh{} y = y.view(seq\PYZus{}len * batch\PYZus{}size, hid\PYZus{}dim)}
        \PY{n}{y} \PY{o}{=} \PY{n+nb+bp}{self}\PY{o}{.}\PY{n}{reg}\PY{p}{(}\PY{n}{y}\PY{p}{)}
        \PY{n}{y} \PY{o}{=} \PY{n}{y}\PY{o}{.}\PY{n}{view}\PY{p}{(}\PY{n}{seq\PYZus{}len}\PY{p}{,} \PY{n}{batch\PYZus{}size}\PY{p}{,} \PY{o}{\PYZhy{}}\PY{l+m+mi}{1}\PY{p}{)} \PY{c+c1}{\PYZsh{} y = y.view(seq\PYZus{}len, batch\PYZus{}size, out\PYZus{}dim)}
        \PY{k}{return} \PY{n}{y}

    \PY{l+s+sd}{\PYZdq{}\PYZdq{}\PYZdq{}}
\PY{l+s+sd}{    PyCharm Crtl+click nn.LSTM() jump to code of PyTorch:}
\PY{l+s+sd}{    Examples::}
\PY{l+s+sd}{        \PYZgt{}\PYZgt{}\PYZgt{} rnn = nn.LSTM(10, 20, 2)}
\PY{l+s+sd}{        \PYZgt{}\PYZgt{}\PYZgt{} input = torch.randn(5, 3, 10)}
\PY{l+s+sd}{        \PYZgt{}\PYZgt{}\PYZgt{} h0 = torch.randn(2, 3, 20)}
\PY{l+s+sd}{        \PYZgt{}\PYZgt{}\PYZgt{} c0 = torch.randn(2, 3, 20)}
\PY{l+s+sd}{        \PYZgt{}\PYZgt{}\PYZgt{} output, (hn, cn) = rnn(input, (h0, c0))}
\PY{l+s+sd}{    \PYZdq{}\PYZdq{}\PYZdq{}}

    \PY{k}{def} \PY{n+nf}{output\PYZus{}y\PYZus{}hc\PYZus{}for\PYZus{}test}\PY{p}{(}\PY{n+nb+bp}{self}\PY{p}{,} \PY{n}{x}\PY{p}{,} \PY{n}{hc}\PY{p}{)}\PY{p}{:}
        \PY{c+c1}{\PYZsh{} 后面计算loss的时候,需要用到y和hc,所以这里需要单独写一个函数}
        \PY{n}{y}\PY{p}{,} \PY{n}{hc} \PY{o}{=} \PY{n+nb+bp}{self}\PY{o}{.}\PY{n}{rnn}\PY{p}{(}\PY{n}{x}\PY{p}{,} \PY{n}{hc}\PY{p}{)}  \PY{c+c1}{\PYZsh{} y, (h, c) = self.rnn(x)}

        \PY{n}{seq\PYZus{}len}\PY{p}{,} \PY{n}{batch\PYZus{}size}\PY{p}{,} \PY{n}{hid\PYZus{}dim} \PY{o}{=} \PY{n}{y}\PY{o}{.}\PY{n}{size}\PY{p}{(}\PY{p}{)}
        \PY{n}{y} \PY{o}{=} \PY{n}{y}\PY{o}{.}\PY{n}{view}\PY{p}{(}\PY{o}{\PYZhy{}}\PY{l+m+mi}{1}\PY{p}{,} \PY{n}{hid\PYZus{}dim}\PY{p}{)}
        \PY{n}{y} \PY{o}{=} \PY{n+nb+bp}{self}\PY{o}{.}\PY{n}{reg}\PY{p}{(}\PY{n}{y}\PY{p}{)}
        \PY{n}{y} \PY{o}{=} \PY{n}{y}\PY{o}{.}\PY{n}{view}\PY{p}{(}\PY{n}{seq\PYZus{}len}\PY{p}{,} \PY{n}{batch\PYZus{}size}\PY{p}{,} \PY{o}{\PYZhy{}}\PY{l+m+mi}{1}\PY{p}{)}
        \PY{k}{return} \PY{n}{y}\PY{p}{,} \PY{n}{hc}
\end{Verbatim}
\end{tcolorbox}

    \hypertarget{ux53c2ux6570ux8bbeux5b9a}{%
\subsubsection{参数设定}\label{ux53c2ux6570ux8bbeux5b9a}}

    \begin{tcolorbox}[breakable, size=fbox, boxrule=1pt, pad at break*=1mm,colback=cellbackground, colframe=cellborder]
\prompt{In}{incolor}{3}{\boxspacing}
\begin{Verbatim}[commandchars=\\\{\}]
\PY{n}{inp\PYZus{}dim} \PY{o}{=} \PY{l+m+mi}{1} \PY{c+c1}{\PYZsh{} 输入维度 我们是(reported result)一个维度}
\PY{n}{out\PYZus{}dim} \PY{o}{=} \PY{l+m+mi}{1} \PY{c+c1}{\PYZsh{} 输出维度 我们是预测客流量,所以是1}
\PY{n}{mid\PYZus{}dim} \PY{o}{=} \PY{l+m+mi}{10} \PY{c+c1}{\PYZsh{} 隐藏层维度}
\PY{n}{mid\PYZus{}layers} \PY{o}{=} \PY{l+m+mi}{1} \PY{c+c1}{\PYZsh{} 隐藏层层数}
\PY{c+c1}{\PYZsh{} batch\PYZus{}size = 12 * 4 \PYZsh{} 我们划分成48个batch \PYZlt{}\PYZhy{}\PYZhy{} 后面改}
\PY{n}{batch\PYZus{}size} \PY{o}{=} \PY{l+m+mi}{100}
\PY{n}{mod\PYZus{}dir} \PY{o}{=} \PY{l+s+s1}{\PYZsq{}}\PY{l+s+s1}{.}\PY{l+s+s1}{\PYZsq{}}
\end{Verbatim}
\end{tcolorbox}

    \hypertarget{load-data}{%
\subsubsection{load data}\label{load-data}}

    \hypertarget{for-data-1}{%
\subparagraph{for data 1}\label{for-data-1}}

    \begin{tcolorbox}[breakable, size=fbox, boxrule=1pt, pad at break*=1mm,colback=cellbackground, colframe=cellborder]
\prompt{In}{incolor}{4}{\boxspacing}
\begin{Verbatim}[commandchars=\\\{\}]
\PY{k+kn}{import} \PY{n+nn}{pandas} \PY{k}{as} \PY{n+nn}{pd}
\PY{c+c1}{\PYZsh{} 读取数据}
\PY{n}{df} \PY{o}{=} \PY{n}{pd}\PY{o}{.}\PY{n}{read\PYZus{}csv}\PY{p}{(}\PY{l+s+s1}{\PYZsq{}}\PY{l+s+s1}{df\PYZus{}Number\PYZus{}of\PYZus{}reported\PYZus{}results.csv}\PY{l+s+s1}{\PYZsq{}}\PY{p}{)}
\PY{n}{seq\PYZus{}number} \PY{o}{=} \PY{n}{df}\PY{o}{.}\PY{n}{values}
\PY{c+c1}{\PYZsh{} 需要先反转}
\PY{n}{seq\PYZus{}number} \PY{o}{=} \PY{n}{seq\PYZus{}number}\PY{p}{[}\PY{p}{:}\PY{p}{:}\PY{o}{\PYZhy{}}\PY{l+m+mi}{1}\PY{p}{]}
\end{Verbatim}
\end{tcolorbox}

    \hypertarget{for-data-2}{%
\subparagraph{for data 2}\label{for-data-2}}

    \begin{tcolorbox}[breakable, size=fbox, boxrule=1pt, pad at break*=1mm,colback=cellbackground, colframe=cellborder]
\prompt{In}{incolor}{5}{\boxspacing}
\begin{Verbatim}[commandchars=\\\{\}]
\PY{c+c1}{\PYZsh{} \PYZsh{} passengers number of international airline , 1949\PYZhy{}01 \PYZti{} 1960\PYZhy{}12 per month}
\PY{c+c1}{\PYZsh{} seq\PYZus{}number = np.array(}
\PY{c+c1}{\PYZsh{}     [112., 118., 132., 129., 121., 135., 148., 148., 136., 119., 104.,}
\PY{c+c1}{\PYZsh{}         118., 115., 126., 141., 135., 125., 149., 170., 170., 158., 133.,}
\PY{c+c1}{\PYZsh{}         114., 140., 145., 150., 178., 163., 172., 178., 199., 199., 184.,}
\PY{c+c1}{\PYZsh{}         162., 146., 166., 171., 180., 193., 181., 183., 218., 230., 242.,}
\PY{c+c1}{\PYZsh{}         209., 191., 172., 194., 196., 196., 236., 235., 229., 243., 264.,}
\PY{c+c1}{\PYZsh{}         272., 237., 211., 180., 201., 204., 188., 235., 227., 234., 264.,}
\PY{c+c1}{\PYZsh{}         302., 293., 259., 229., 203., 229., 242., 233., 267., 269., 270.,}
\PY{c+c1}{\PYZsh{}         315., 364., 347., 312., 274., 237., 278., 284., 277., 317., 313.,}
\PY{c+c1}{\PYZsh{}         318., 374., 413., 405., 355., 306., 271., 306., 315., 301., 356.,}
\PY{c+c1}{\PYZsh{}         348., 355., 422., 465., 467., 404., 347., 305., 336., 340., 318.,}
\PY{c+c1}{\PYZsh{}         362., 348., 363., 435., 491., 505., 404., 359., 310., 337., 360.,}
\PY{c+c1}{\PYZsh{}         342., 406., 396., 420., 472., 548., 559., 463., 407., 362., 405.,}
\PY{c+c1}{\PYZsh{}         417., 391., 419., 461., 472., 535., 622., 606., 508., 461., 390.,}
\PY{c+c1}{\PYZsh{}         432.], dtype=np.float32)}
\PY{c+c1}{\PYZsh{} \PYZsh{} 给seq\PYZus{}number增加一个维度,变成2维的}
\PY{c+c1}{\PYZsh{} seq\PYZus{}number = seq\PYZus{}number[:, np.newaxis]}
\end{Verbatim}
\end{tcolorbox}

    \hypertarget{show-data}{%
\subparagraph{show data}\label{show-data}}

    \begin{tcolorbox}[breakable, size=fbox, boxrule=1pt, pad at break*=1mm,colback=cellbackground, colframe=cellborder]
\prompt{In}{incolor}{6}{\boxspacing}
\begin{Verbatim}[commandchars=\\\{\}]
\PY{n}{plt}\PY{o}{.}\PY{n}{figure}\PY{p}{(}\PY{n}{figsize}\PY{o}{=}\PY{p}{(}\PY{l+m+mi}{12}\PY{p}{,} \PY{l+m+mi}{6}\PY{p}{)}\PY{p}{)}
\PY{n}{plt}\PY{o}{.}\PY{n}{plot}\PY{p}{(}\PY{n}{seq\PYZus{}number}\PY{p}{)}
\end{Verbatim}
\end{tcolorbox}

            \begin{tcolorbox}[breakable, size=fbox, boxrule=.5pt, pad at break*=1mm, opacityfill=0]
\prompt{Out}{outcolor}{6}{\boxspacing}
\begin{Verbatim}[commandchars=\\\{\}]
[<matplotlib.lines.Line2D at 0x1d1a6108dc0>]
\end{Verbatim}
\end{tcolorbox}
        
    \begin{center}
    \adjustimage{max size={0.9\linewidth}{0.9\paperheight}}{output_13_1.png}
    \end{center}
    { \hspace*{\fill} \\}
    
    \begin{tcolorbox}[breakable, size=fbox, boxrule=1pt, pad at break*=1mm,colback=cellbackground, colframe=cellborder]
\prompt{In}{incolor}{7}{\boxspacing}
\begin{Verbatim}[commandchars=\\\{\}]
\PY{n}{seq\PYZus{}number}\PY{o}{.}\PY{n}{shape} \PY{c+c1}{\PYZsh{} 1月17号到12月31号的数据,共359天}
\end{Verbatim}
\end{tcolorbox}

            \begin{tcolorbox}[breakable, size=fbox, boxrule=.5pt, pad at break*=1mm, opacityfill=0]
\prompt{Out}{outcolor}{7}{\boxspacing}
\begin{Verbatim}[commandchars=\\\{\}]
(359, 1)
\end{Verbatim}
\end{tcolorbox}
        
    \begin{tcolorbox}[breakable, size=fbox, boxrule=1pt, pad at break*=1mm,colback=cellbackground, colframe=cellborder]
\prompt{In}{incolor}{8}{\boxspacing}
\begin{Verbatim}[commandchars=\\\{\}]
\PY{n}{seq} \PY{o}{=} \PY{n}{seq\PYZus{}number} \PY{c+c1}{\PYZsh{} 所以最终的形式是(当天reported result) }
\end{Verbatim}
\end{tcolorbox}

    \begin{tcolorbox}[breakable, size=fbox, boxrule=1pt, pad at break*=1mm,colback=cellbackground, colframe=cellborder]
\prompt{In}{incolor}{9}{\boxspacing}
\begin{Verbatim}[commandchars=\\\{\}]
\PY{c+c1}{\PYZsh{} 转化成浮点数}
\PY{n}{seq} \PY{o}{=} \PY{n}{seq}\PY{o}{.}\PY{n}{astype}\PY{p}{(}\PY{n}{np}\PY{o}{.}\PY{n}{float32}\PY{p}{)}
\PY{n}{seq}\PY{p}{[}\PY{p}{:}\PY{l+m+mi}{5}\PY{p}{]}
\end{Verbatim}
\end{tcolorbox}

            \begin{tcolorbox}[breakable, size=fbox, boxrule=.5pt, pad at break*=1mm, opacityfill=0]
\prompt{Out}{outcolor}{9}{\boxspacing}
\begin{Verbatim}[commandchars=\\\{\}]
array([[ 80630.],
       [101503.],
       [ 91477.],
       [107134.],
       [153880.]], dtype=float32)
\end{Verbatim}
\end{tcolorbox}
        
    \begin{tcolorbox}[breakable, size=fbox, boxrule=1pt, pad at break*=1mm,colback=cellbackground, colframe=cellborder]
\prompt{In}{incolor}{10}{\boxspacing}
\begin{Verbatim}[commandchars=\\\{\}]
\PY{c+c1}{\PYZsh{} normalization}
\PY{n}{seq\PYZus{}mean} \PY{o}{=} \PY{n}{seq}\PY{o}{.}\PY{n}{mean}\PY{p}{(}\PY{n}{axis}\PY{o}{=}\PY{l+m+mi}{0}\PY{p}{)}
\PY{n}{seq\PYZus{}std} \PY{o}{=} \PY{n}{seq}\PY{o}{.}\PY{n}{std}\PY{p}{(}\PY{n}{axis}\PY{o}{=}\PY{l+m+mi}{0}\PY{p}{)}
\PY{n}{seq\PYZus{}norm} \PY{o}{=} \PY{p}{(}\PY{n}{seq} \PY{o}{\PYZhy{}} \PY{n}{seq\PYZus{}mean}\PY{p}{)} \PY{o}{/} \PY{n}{seq\PYZus{}std}
\end{Verbatim}
\end{tcolorbox}

    \begin{tcolorbox}[breakable, size=fbox, boxrule=1pt, pad at break*=1mm,colback=cellbackground, colframe=cellborder]
\prompt{In}{incolor}{11}{\boxspacing}
\begin{Verbatim}[commandchars=\\\{\}]
\PY{n}{seq\PYZus{}norm}\PY{p}{[}\PY{p}{:}\PY{l+m+mi}{5}\PY{p}{]}
\end{Verbatim}
\end{tcolorbox}

            \begin{tcolorbox}[breakable, size=fbox, boxrule=.5pt, pad at break*=1mm, opacityfill=0]
\prompt{Out}{outcolor}{11}{\boxspacing}
\begin{Verbatim}[commandchars=\\\{\}]
array([[-0.11607783],
       [ 0.11818092],
       [ 0.00565861],
       [ 0.18137792],
       [ 0.7060107 ]], dtype=float32)
\end{Verbatim}
\end{tcolorbox}
        
    \begin{tcolorbox}[breakable, size=fbox, boxrule=1pt, pad at break*=1mm,colback=cellbackground, colframe=cellborder]
\prompt{In}{incolor}{12}{\boxspacing}
\begin{Verbatim}[commandchars=\\\{\}]
\PY{n}{data} \PY{o}{=} \PY{n}{seq\PYZus{}norm} 
\PY{n}{data\PYZus{}x} \PY{o}{=} \PY{n}{data}\PY{p}{[}\PY{p}{:}\PY{o}{\PYZhy{}}\PY{l+m+mi}{1}\PY{p}{,} \PY{p}{:}\PY{p}{]} \PY{c+c1}{\PYZsh{} 从0到倒数第二个}
\PY{n}{data\PYZus{}y} \PY{o}{=} \PY{n}{data}\PY{p}{[}\PY{o}{+}\PY{l+m+mi}{1}\PY{p}{:}\PY{p}{,} \PY{l+m+mi}{0}\PY{p}{]} \PY{c+c1}{\PYZsh{} 从1到最后一个}
\PY{k}{assert} \PY{n}{data\PYZus{}x}\PY{o}{.}\PY{n}{shape}\PY{p}{[}\PY{l+m+mi}{1}\PY{p}{]} \PY{o}{==} \PY{n}{inp\PYZus{}dim}
\PY{n+nb}{print}\PY{p}{(}\PY{l+s+s2}{\PYZdq{}}\PY{l+s+s2}{data\PYZus{}x[:5]:}\PY{l+s+s2}{\PYZdq{}}\PY{p}{,} \PY{n}{data\PYZus{}x}\PY{p}{[}\PY{p}{:}\PY{l+m+mi}{5}\PY{p}{]}\PY{p}{)}
\PY{n+nb}{print}\PY{p}{(}\PY{l+s+s2}{\PYZdq{}}\PY{l+s+s2}{data\PYZus{}y[:5]:}\PY{l+s+s2}{\PYZdq{}}\PY{p}{,} \PY{n}{data\PYZus{}y}\PY{p}{[}\PY{p}{:}\PY{l+m+mi}{5}\PY{p}{]}\PY{p}{)}
\PY{n+nb}{print}\PY{p}{(}\PY{l+s+s2}{\PYZdq{}}\PY{l+s+s2}{data\PYZus{}x.shape:}\PY{l+s+s2}{\PYZdq{}}\PY{p}{,} \PY{n}{data\PYZus{}x}\PY{o}{.}\PY{n}{shape}\PY{p}{)}
\PY{n+nb}{print}\PY{p}{(}\PY{l+s+s2}{\PYZdq{}}\PY{l+s+s2}{data\PYZus{}y.shape:}\PY{l+s+s2}{\PYZdq{}}\PY{p}{,} \PY{n}{data\PYZus{}y}\PY{o}{.}\PY{n}{shape}\PY{p}{)}
\end{Verbatim}
\end{tcolorbox}

    \begin{Verbatim}[commandchars=\\\{\}]
data\_x[:5]: [[-0.11607783]
 [ 0.11818092]
 [ 0.00565861]
 [ 0.18137792]
 [ 0.7060107 ]]
data\_y[:5]: [0.11818092 0.00565861 0.18137792 0.7060107  0.5231423 ]
data\_x.shape: (358, 1)
data\_y.shape: (358,)
    \end{Verbatim}

    \begin{tcolorbox}[breakable, size=fbox, boxrule=1pt, pad at break*=1mm,colback=cellbackground, colframe=cellborder]
\prompt{In}{incolor}{13}{\boxspacing}
\begin{Verbatim}[commandchars=\\\{\}]
\PY{c+c1}{\PYZsh{}\PYZsh{} 2022.2.17到2022.10.29 都是训练集,2022.10.30到2022.12.31都是测试集 也就是一共 497 \PYZhy{} 202 + 1 = 296天 }
\PY{c+c1}{\PYZsh{} 296/len(data\PYZus{}x) \PYZsh{} \PYZlt{}\PYZhy{}\PYZhy{} for data1}
\end{Verbatim}
\end{tcolorbox}

    \begin{tcolorbox}[breakable, size=fbox, boxrule=1pt, pad at break*=1mm,colback=cellbackground, colframe=cellborder]
\prompt{In}{incolor}{14}{\boxspacing}
\begin{Verbatim}[commandchars=\\\{\}]
\PY{c+c1}{\PYZsh{} split train and test}
\PY{c+c1}{\PYZsh{} train\PYZus{}size = 296 \PYZsh{} for data1}
\PY{n}{train\PYZus{}size} \PY{o}{=} \PY{n+nb}{int}\PY{p}{(}\PY{n+nb}{len}\PY{p}{(}\PY{n}{data\PYZus{}x}\PY{p}{)} \PY{o}{*} \PY{l+m+mf}{0.70}\PY{p}{)} \PY{c+c1}{\PYZsh{} 75\PYZpc{}的数据作为训练集 for data2}

\PY{n}{train\PYZus{}x} \PY{o}{=} \PY{n}{data\PYZus{}x}\PY{p}{[}\PY{p}{:}\PY{n}{train\PYZus{}size}\PY{p}{]}
\PY{n}{train\PYZus{}y} \PY{o}{=} \PY{n}{data\PYZus{}y}\PY{p}{[}\PY{p}{:}\PY{n}{train\PYZus{}size}\PY{p}{]}

\PY{n}{train\PYZus{}x} \PY{o}{=} \PY{n}{train\PYZus{}x}\PY{o}{.}\PY{n}{reshape}\PY{p}{(}\PY{p}{(}\PY{n}{train\PYZus{}size}\PY{p}{,} \PY{n}{inp\PYZus{}dim}\PY{p}{)}\PY{p}{)} \PY{c+c1}{\PYZsh{} 296, 1}
\PY{n}{train\PYZus{}y} \PY{o}{=} \PY{n}{train\PYZus{}y}\PY{o}{.}\PY{n}{reshape}\PY{p}{(}\PY{p}{(}\PY{n}{train\PYZus{}size}\PY{p}{,} \PY{n}{out\PYZus{}dim}\PY{p}{)}\PY{p}{)} \PY{c+c1}{\PYZsh{} 296, 1}
\PY{n+nb}{print}\PY{p}{(}\PY{l+s+s2}{\PYZdq{}}\PY{l+s+s2}{train\PYZus{}x.shape: }\PY{l+s+s2}{\PYZdq{}}\PY{p}{,} \PY{n}{train\PYZus{}x}\PY{o}{.}\PY{n}{shape}\PY{p}{)}
\PY{n+nb}{print}\PY{p}{(}\PY{l+s+s2}{\PYZdq{}}\PY{l+s+s2}{train\PYZus{}y.shape: }\PY{l+s+s2}{\PYZdq{}}\PY{p}{,} \PY{n}{train\PYZus{}y}\PY{o}{.}\PY{n}{shape}\PY{p}{)}
\end{Verbatim}
\end{tcolorbox}

    \begin{Verbatim}[commandchars=\\\{\}]
train\_x.shape:  (250, 1)
train\_y.shape:  (250, 1)
    \end{Verbatim}

    \hypertarget{build-model}{%
\subsubsection{build model}\label{build-model}}

    \begin{tcolorbox}[breakable, size=fbox, boxrule=1pt, pad at break*=1mm,colback=cellbackground, colframe=cellborder]
\prompt{In}{incolor}{15}{\boxspacing}
\begin{Verbatim}[commandchars=\\\{\}]
\PY{n}{device} \PY{o}{=} \PY{n}{torch}\PY{o}{.}\PY{n}{device}\PY{p}{(}\PY{l+s+s2}{\PYZdq{}}\PY{l+s+s2}{cuda}\PY{l+s+s2}{\PYZdq{}} \PY{k}{if} \PY{n}{torch}\PY{o}{.}\PY{n}{cuda}\PY{o}{.}\PY{n}{is\PYZus{}available}\PY{p}{(}\PY{p}{)} \PY{k}{else} \PY{l+s+s2}{\PYZdq{}}\PY{l+s+s2}{cpu}\PY{l+s+s2}{\PYZdq{}}\PY{p}{)}
\PY{n}{net} \PY{o}{=} \PY{n}{RegLSTM}\PY{p}{(}\PY{n}{inp\PYZus{}dim}\PY{p}{,} \PY{n}{out\PYZus{}dim}\PY{p}{,} \PY{n}{mid\PYZus{}dim}\PY{p}{,} \PY{n}{mid\PYZus{}layers}\PY{p}{)}\PY{o}{.}\PY{n}{to}\PY{p}{(}\PY{n}{device}\PY{p}{)}
\PY{n}{criterion} \PY{o}{=} \PY{n}{nn}\PY{o}{.}\PY{n}{MSELoss}\PY{p}{(}\PY{p}{)}
\PY{n}{optimizer} \PY{o}{=} \PY{n}{torch}\PY{o}{.}\PY{n}{optim}\PY{o}{.}\PY{n}{Adam}\PY{p}{(}\PY{n}{net}\PY{o}{.}\PY{n}{parameters}\PY{p}{(}\PY{p}{)}\PY{p}{,} \PY{n}{lr}\PY{o}{=}\PY{l+m+mf}{1e\PYZhy{}2}\PY{p}{)}
\end{Verbatim}
\end{tcolorbox}

    \begin{tcolorbox}[breakable, size=fbox, boxrule=1pt, pad at break*=1mm,colback=cellbackground, colframe=cellborder]
\prompt{In}{incolor}{16}{\boxspacing}
\begin{Verbatim}[commandchars=\\\{\}]
\PY{k+kn}{from} \PY{n+nn}{torchinfo} \PY{k+kn}{import} \PY{n}{summary}
\PY{n}{summary}\PY{p}{(}\PY{n}{net}\PY{p}{,} \PY{n}{input\PYZus{}size}\PY{o}{=}\PY{p}{(}\PY{n}{batch\PYZus{}size} \PY{o}{\PYZhy{}} \PY{l+m+mi}{1}\PY{p}{,} \PY{n}{batch\PYZus{}size}\PY{p}{,} \PY{l+m+mi}{1}\PY{p}{)}\PY{p}{)}
\end{Verbatim}
\end{tcolorbox}

            \begin{tcolorbox}[breakable, size=fbox, boxrule=.5pt, pad at break*=1mm, opacityfill=0]
\prompt{Out}{outcolor}{16}{\boxspacing}
\begin{Verbatim}[commandchars=\\\{\}]
================================================================================
==========
Layer (type:depth-idx)                   Output Shape              Param \#
================================================================================
==========
RegLSTM                                  [99, 100, 1]              --
├─LSTM: 1-1                              [99, 100, 10]             520
├─Sequential: 1-2                        [9900, 1]                 --
│    └─Linear: 2-1                       [9900, 10]                110
│    └─Tanh: 2-2                         [9900, 10]                --
│    └─Linear: 2-3                       [9900, 1]                 11
================================================================================
==========
Total params: 641
Trainable params: 641
Non-trainable params: 0
Total mult-adds (M): 6.35
================================================================================
==========
Input size (MB): 0.04
Forward/backward pass size (MB): 1.66
Params size (MB): 0.00
Estimated Total Size (MB): 1.71
================================================================================
==========
\end{Verbatim}
\end{tcolorbox}
        
    \hypertarget{train}{%
\subsubsection{train}\label{train}}

    \hypertarget{ux5236ux4f5cbatch}{%
\paragraph{制作batch}\label{ux5236ux4f5cbatch}}

\begin{center}
	\adjustimage{max size={0.7\linewidth}{0.7\paperheight}}{lstm_batch.drawio.png}
\end{center}

我们首先制作train的batch,然后训练模型。

制作batch的方法是选取不同开头但截止一样的序列,然后将这些序列组合成一个batch。

    \hypertarget{ux5236ux4f5cux4e00ux4e2abatch}{%
\subparagraph{制作一个batch}\label{ux5236ux4f5cux4e00ux4e2abatch}}

    \begin{tcolorbox}[breakable, size=fbox, boxrule=1pt, pad at break*=1mm,colback=cellbackground, colframe=cellborder]
\prompt{In}{incolor}{17}{\boxspacing}
\begin{Verbatim}[commandchars=\\\{\}]
\PY{c+c1}{\PYZsh{} var\PYZus{}x = torch.tensor(train\PYZus{}x, dtype=torch.float32, device=device)}
\PY{c+c1}{\PYZsh{} var\PYZus{}y = torch.tensor(train\PYZus{}y, dtype=torch.float32, device=device)}

\PY{c+c1}{\PYZsh{} batch\PYZus{}var\PYZus{}x = list()}
\PY{c+c1}{\PYZsh{} batch\PYZus{}var\PYZus{}y = list()}

\PY{c+c1}{\PYZsh{} for roi\PYZus{}len in range(batch\PYZus{}size):}
\PY{c+c1}{\PYZsh{}     begin\PYZus{}idx = train\PYZus{}size \PYZhy{} roi\PYZus{}len \PYZsh{} train\PYZus{}size = 296}
\PY{c+c1}{\PYZsh{}     batch\PYZus{}var\PYZus{}x.append(var\PYZus{}x[begin\PYZus{}idx:])}
\PY{c+c1}{\PYZsh{}     batch\PYZus{}var\PYZus{}y.append(var\PYZus{}y[begin\PYZus{}idx:])}
\end{Verbatim}
\end{tcolorbox}

    \begin{tcolorbox}[breakable, size=fbox, boxrule=1pt, pad at break*=1mm,colback=cellbackground, colframe=cellborder]
\prompt{In}{incolor}{18}{\boxspacing}
\begin{Verbatim}[commandchars=\\\{\}]
\PY{c+c1}{\PYZsh{} print(\PYZdq{}var\PYZus{}x.shape: \PYZdq{}, var\PYZus{}x.shape)}
\PY{c+c1}{\PYZsh{} print(\PYZdq{}var\PYZus{}y.shape: \PYZdq{}, var\PYZus{}y.shape)}
\PY{c+c1}{\PYZsh{} print(\PYZdq{}batch\PYZus{}var\PYZus{}x[0].shape: \PYZdq{}, batch\PYZus{}var\PYZus{}x[0].shape)}
\PY{c+c1}{\PYZsh{} print(\PYZdq{}batch\PYZus{}var\PYZus{}y[0].shape: \PYZdq{}, batch\PYZus{}var\PYZus{}y[0].shape)}
\PY{c+c1}{\PYZsh{} print(\PYZdq{}batch\PYZus{}var\PYZus{}x[1].shape: \PYZdq{}, batch\PYZus{}var\PYZus{}x[\PYZhy{}1].shape)}
\PY{c+c1}{\PYZsh{} print(\PYZdq{}batch\PYZus{}var\PYZus{}y[1].shape: \PYZdq{}, batch\PYZus{}var\PYZus{}y[\PYZhy{}1].shape)}
\PY{c+c1}{\PYZsh{} print(\PYZdq{}batch\PYZus{}var\PYZus{}x.len: \PYZdq{}, len(batch\PYZus{}var\PYZus{}x))}
\PY{c+c1}{\PYZsh{} print(\PYZdq{}batch\PYZus{}var\PYZus{}y.len: \PYZdq{}, len(batch\PYZus{}var\PYZus{}y))}
\end{Verbatim}
\end{tcolorbox}

    \hypertarget{ux5236ux4f5cux591aux4e2amini_batchux4f7fux5f97ux4e00ux4e2aepochux53efux4ee5ux770bux5b8cux6240ux6709ux6570ux636e}{%
\subparagraph{制作多个mini\_batch使得一个epoch可以看完所有数据}\label{ux5236ux4f5cux591aux4e2amini_batchux4f7fux5f97ux4e00ux4e2aepochux53efux4ee5ux770bux5b8cux6240ux6709ux6570ux636e}}

    \begin{tcolorbox}[breakable, size=fbox, boxrule=1pt, pad at break*=1mm,colback=cellbackground, colframe=cellborder]
\prompt{In}{incolor}{19}{\boxspacing}
\begin{Verbatim}[commandchars=\\\{\}]
\PY{n}{batch\PYZus{}size}
\end{Verbatim}
\end{tcolorbox}

            \begin{tcolorbox}[breakable, size=fbox, boxrule=.5pt, pad at break*=1mm, opacityfill=0]
\prompt{Out}{outcolor}{19}{\boxspacing}
\begin{Verbatim}[commandchars=\\\{\}]
100
\end{Verbatim}
\end{tcolorbox}
        
    \begin{tcolorbox}[breakable, size=fbox, boxrule=1pt, pad at break*=1mm,colback=cellbackground, colframe=cellborder]
\prompt{In}{incolor}{20}{\boxspacing}
\begin{Verbatim}[commandchars=\\\{\}]
\PY{n}{train\PYZus{}size}
\end{Verbatim}
\end{tcolorbox}

            \begin{tcolorbox}[breakable, size=fbox, boxrule=.5pt, pad at break*=1mm, opacityfill=0]
\prompt{Out}{outcolor}{20}{\boxspacing}
\begin{Verbatim}[commandchars=\\\{\}]
250
\end{Verbatim}
\end{tcolorbox}
        
    \begin{tcolorbox}[breakable, size=fbox, boxrule=1pt, pad at break*=1mm,colback=cellbackground, colframe=cellborder]
\prompt{In}{incolor}{21}{\boxspacing}
\begin{Verbatim}[commandchars=\\\{\}]
\PY{c+c1}{\PYZsh{} 那么我们大概需要再制作 4个batch就好了,重复是必须的}
\PY{n}{var\PYZus{}x} \PY{o}{=} \PY{n}{torch}\PY{o}{.}\PY{n}{tensor}\PY{p}{(}\PY{n}{train\PYZus{}x}\PY{p}{,} \PY{n}{dtype}\PY{o}{=}\PY{n}{torch}\PY{o}{.}\PY{n}{float32}\PY{p}{,} \PY{n}{device}\PY{o}{=}\PY{n}{device}\PY{p}{)}
\PY{n}{var\PYZus{}y} \PY{o}{=} \PY{n}{torch}\PY{o}{.}\PY{n}{tensor}\PY{p}{(}\PY{n}{train\PYZus{}y}\PY{p}{,} \PY{n}{dtype}\PY{o}{=}\PY{n}{torch}\PY{o}{.}\PY{n}{float32}\PY{p}{,} \PY{n}{device}\PY{o}{=}\PY{n}{device}\PY{p}{)}

\PY{n}{batch\PYZus{}var\PYZus{}x} \PY{o}{=} \PY{n+nb}{list}\PY{p}{(}\PY{p}{)}
\PY{n}{batch\PYZus{}var\PYZus{}y} \PY{o}{=} \PY{n+nb}{list}\PY{p}{(}\PY{p}{)}

\PY{n}{mini\PYZus{}batch\PYZus{}size} \PY{o}{=} \PY{l+m+mi}{5}

\PY{k}{for} \PY{n}{num\PYZus{}of\PYZus{}batch} \PY{o+ow}{in} \PY{n+nb}{range}\PY{p}{(}\PY{n}{mini\PYZus{}batch\PYZus{}size}\PY{p}{)}\PY{p}{:}
    \PY{n}{end\PYZus{}idx} \PY{o}{=} \PY{n}{np}\PY{o}{.}\PY{n}{random}\PY{o}{.}\PY{n}{randint}\PY{p}{(}\PY{n}{batch\PYZus{}size}\PY{p}{,} \PY{n}{train\PYZus{}size}\PY{p}{)}
    \PY{k}{for} \PY{n}{roi\PYZus{}len} \PY{o+ow}{in} \PY{n+nb}{range}\PY{p}{(}\PY{n}{batch\PYZus{}size}\PY{p}{)}\PY{p}{:}
        \PY{n}{begin\PYZus{}idx} \PY{o}{=} \PY{n}{end\PYZus{}idx} \PY{o}{\PYZhy{}} \PY{n}{roi\PYZus{}len} 
        \PY{n}{batch\PYZus{}var\PYZus{}x}\PY{o}{.}\PY{n}{append}\PY{p}{(}\PY{n}{var\PYZus{}x}\PY{p}{[}\PY{n}{begin\PYZus{}idx}\PY{p}{:}\PY{n}{end\PYZus{}idx}\PY{p}{]}\PY{p}{)}
        \PY{n}{batch\PYZus{}var\PYZus{}y}\PY{o}{.}\PY{n}{append}\PY{p}{(}\PY{n}{var\PYZus{}y}\PY{p}{[}\PY{n}{begin\PYZus{}idx}\PY{p}{:}\PY{n}{end\PYZus{}idx}\PY{p}{]}\PY{p}{)}
\end{Verbatim}
\end{tcolorbox}

    \begin{tcolorbox}[breakable, size=fbox, boxrule=1pt, pad at break*=1mm,colback=cellbackground, colframe=cellborder]
\prompt{In}{incolor}{22}{\boxspacing}
\begin{Verbatim}[commandchars=\\\{\}]
\PY{n+nb}{print}\PY{p}{(}\PY{l+s+s2}{\PYZdq{}}\PY{l+s+s2}{var\PYZus{}x.shape: }\PY{l+s+s2}{\PYZdq{}}\PY{p}{,} \PY{n}{var\PYZus{}x}\PY{o}{.}\PY{n}{shape}\PY{p}{)}
\PY{n+nb}{print}\PY{p}{(}\PY{l+s+s2}{\PYZdq{}}\PY{l+s+s2}{var\PYZus{}y.shape: }\PY{l+s+s2}{\PYZdq{}}\PY{p}{,} \PY{n}{var\PYZus{}y}\PY{o}{.}\PY{n}{shape}\PY{p}{)}
\PY{n+nb}{print}\PY{p}{(}\PY{l+s+s2}{\PYZdq{}}\PY{l+s+s2}{batch\PYZus{}var\PYZus{}x[0].shape: }\PY{l+s+s2}{\PYZdq{}}\PY{p}{,} \PY{n}{batch\PYZus{}var\PYZus{}x}\PY{p}{[}\PY{l+m+mi}{0}\PY{p}{]}\PY{o}{.}\PY{n}{shape}\PY{p}{)}
\PY{n+nb}{print}\PY{p}{(}\PY{l+s+s2}{\PYZdq{}}\PY{l+s+s2}{batch\PYZus{}var\PYZus{}y[0].shape: }\PY{l+s+s2}{\PYZdq{}}\PY{p}{,} \PY{n}{batch\PYZus{}var\PYZus{}y}\PY{p}{[}\PY{l+m+mi}{0}\PY{p}{]}\PY{o}{.}\PY{n}{shape}\PY{p}{)}
\PY{n+nb}{print}\PY{p}{(}\PY{l+s+s2}{\PYZdq{}}\PY{l+s+s2}{batch\PYZus{}var\PYZus{}x[\PYZhy{}1].shape: }\PY{l+s+s2}{\PYZdq{}}\PY{p}{,} \PY{n}{batch\PYZus{}var\PYZus{}x}\PY{p}{[}\PY{o}{\PYZhy{}}\PY{l+m+mi}{1}\PY{p}{]}\PY{o}{.}\PY{n}{shape}\PY{p}{)}
\PY{n+nb}{print}\PY{p}{(}\PY{l+s+s2}{\PYZdq{}}\PY{l+s+s2}{batch\PYZus{}var\PYZus{}y[\PYZhy{}1].shape: }\PY{l+s+s2}{\PYZdq{}}\PY{p}{,} \PY{n}{batch\PYZus{}var\PYZus{}y}\PY{p}{[}\PY{o}{\PYZhy{}}\PY{l+m+mi}{1}\PY{p}{]}\PY{o}{.}\PY{n}{shape}\PY{p}{)}
\PY{n+nb}{print}\PY{p}{(}\PY{l+s+s2}{\PYZdq{}}\PY{l+s+s2}{batch\PYZus{}var\PYZus{}x.len: }\PY{l+s+s2}{\PYZdq{}}\PY{p}{,} \PY{n+nb}{len}\PY{p}{(}\PY{n}{batch\PYZus{}var\PYZus{}x}\PY{p}{)}\PY{p}{)}
\PY{n+nb}{print}\PY{p}{(}\PY{l+s+s2}{\PYZdq{}}\PY{l+s+s2}{batch\PYZus{}var\PYZus{}y.len: }\PY{l+s+s2}{\PYZdq{}}\PY{p}{,} \PY{n+nb}{len}\PY{p}{(}\PY{n}{batch\PYZus{}var\PYZus{}y}\PY{p}{)}\PY{p}{)}
\end{Verbatim}
\end{tcolorbox}

    \begin{Verbatim}[commandchars=\\\{\}]
var\_x.shape:  torch.Size([250, 1])
var\_y.shape:  torch.Size([250, 1])
batch\_var\_x[0].shape:  torch.Size([0, 1])
batch\_var\_y[0].shape:  torch.Size([0, 1])
batch\_var\_x[-1].shape:  torch.Size([99, 1])
batch\_var\_y[-1].shape:  torch.Size([99, 1])
batch\_var\_x.len:  500
batch\_var\_y.len:  500
    \end{Verbatim}

    \hypertarget{padding}{%
\paragraph{padding}\label{padding}}

\begin{center}
	\adjustimage{max size={0.5\linewidth}{0.5\paperheight}}{lstm.drawio.png}
\end{center}

我们看到不同的batch的shape是不一样的,但是我们需要的是一样的,所以我们需要对batch进行padding,即在后面补0,使得所有的batch的shape都一样

    \begin{tcolorbox}[breakable, size=fbox, boxrule=1pt, pad at break*=1mm,colback=cellbackground, colframe=cellborder]
\prompt{In}{incolor}{23}{\boxspacing}
\begin{Verbatim}[commandchars=\\\{\}]
\PY{k+kn}{from} \PY{n+nn}{torch}\PY{n+nn}{.}\PY{n+nn}{nn}\PY{n+nn}{.}\PY{n+nn}{utils}\PY{n+nn}{.}\PY{n+nn}{rnn} \PY{k+kn}{import} \PY{n}{pad\PYZus{}sequence}
\PY{n}{batch\PYZus{}var\PYZus{}x} \PY{o}{=} \PY{n}{pad\PYZus{}sequence}\PY{p}{(}\PY{n}{batch\PYZus{}var\PYZus{}x}\PY{p}{)}
\PY{n}{batch\PYZus{}var\PYZus{}y} \PY{o}{=} \PY{n}{pad\PYZus{}sequence}\PY{p}{(}\PY{n}{batch\PYZus{}var\PYZus{}y}\PY{p}{)}
\PY{n+nb}{print}\PY{p}{(}\PY{l+s+s2}{\PYZdq{}}\PY{l+s+s2}{batch\PYZus{}var\PYZus{}x.shape: }\PY{l+s+s2}{\PYZdq{}}\PY{p}{,} \PY{n}{batch\PYZus{}var\PYZus{}x}\PY{o}{.}\PY{n}{shape}\PY{p}{)}
\PY{n+nb}{print}\PY{p}{(}\PY{l+s+s2}{\PYZdq{}}\PY{l+s+s2}{batch\PYZus{}var\PYZus{}y.shape: }\PY{l+s+s2}{\PYZdq{}}\PY{p}{,} \PY{n}{batch\PYZus{}var\PYZus{}y}\PY{o}{.}\PY{n}{shape}\PY{p}{)}
\end{Verbatim}
\end{tcolorbox}

    \begin{Verbatim}[commandchars=\\\{\}]
batch\_var\_x.shape:  torch.Size([99, 500, 1])
batch\_var\_y.shape:  torch.Size([99, 500, 1])
    \end{Verbatim}

    \begin{tcolorbox}[breakable, size=fbox, boxrule=1pt, pad at break*=1mm,colback=cellbackground, colframe=cellborder]
\prompt{In}{incolor}{24}{\boxspacing}
\begin{Verbatim}[commandchars=\\\{\}]
\PY{n}{batch\PYZus{}var\PYZus{}x}\PY{p}{[}\PY{p}{:}\PY{l+m+mi}{5}\PY{p}{,} \PY{l+m+mi}{0}\PY{p}{,} \PY{p}{:}\PY{p}{]} \PY{c+c1}{\PYZsh{} 对应上图的蓝色圆角矩形上方第一行数据前五个}
\end{Verbatim}
\end{tcolorbox}

            \begin{tcolorbox}[breakable, size=fbox, boxrule=.5pt, pad at break*=1mm, opacityfill=0]
\prompt{Out}{outcolor}{24}{\boxspacing}
\begin{Verbatim}[commandchars=\\\{\}]
tensor([[0.],
        [0.],
        [0.],
        [0.],
        [0.]])
\end{Verbatim}
\end{tcolorbox}
        
    \begin{tcolorbox}[breakable, size=fbox, boxrule=1pt, pad at break*=1mm,colback=cellbackground, colframe=cellborder]
\prompt{In}{incolor}{25}{\boxspacing}
\begin{Verbatim}[commandchars=\\\{\}]
\PY{n}{batch\PYZus{}var\PYZus{}x}\PY{p}{[}\PY{p}{:}\PY{l+m+mi}{5}\PY{p}{,} \PY{l+m+mi}{1}\PY{p}{,} \PY{p}{:}\PY{p}{]}  \PY{c+c1}{\PYZsh{} 对应上图的蓝色圆角矩形所在行数据前五个}
\end{Verbatim}
\end{tcolorbox}

            \begin{tcolorbox}[breakable, size=fbox, boxrule=.5pt, pad at break*=1mm, opacityfill=0]
\prompt{Out}{outcolor}{25}{\boxspacing}
\begin{Verbatim}[commandchars=\\\{\}]
tensor([[-0.0579],
        [ 0.0000],
        [ 0.0000],
        [ 0.0000],
        [ 0.0000]])
\end{Verbatim}
\end{tcolorbox}
        
    \hypertarget{ux9057ux5fd8ux66f2ux7ebfux63a7ux5236loss}{%
\paragraph{遗忘曲线控制loss}\label{ux9057ux5fd8ux66f2ux7ebfux63a7ux5236loss}}

这里为损失函数添加了类似遗忘曲线的东西,使得模型在训练的时候不会忘记之前的信息,而是会逐渐遗忘。

\[
weight1_{t} = tanh(e * \frac{t}{len(train_y)}) \quad \text{where} \quad t \in [0, len(train_y))
\]

试图改一下遗忘曲线,使得模型在训练的时候对近期记忆保持更多的记忆

\[
weight2_{t} = tanh\left (\alpha * \left (\frac{t}{len(train_y)}-1 \right ) \right )+1 \quad \text{where} \quad t \in [0, len(train_y))
\]

    \begin{tcolorbox}[breakable, size=fbox, boxrule=1pt, pad at break*=1mm,colback=cellbackground, colframe=cellborder]
\prompt{In}{incolor}{26}{\boxspacing}
\begin{Verbatim}[commandchars=\\\{\}]
\PY{k}{with} \PY{n}{torch}\PY{o}{.}\PY{n}{no\PYZus{}grad}\PY{p}{(}\PY{p}{)}\PY{p}{:}
    \PY{n}{weights} \PY{o}{=} \PY{n}{np}\PY{o}{.}\PY{n}{tanh}\PY{p}{(}\PY{n}{np}\PY{o}{.}\PY{n}{arange}\PY{p}{(}\PY{n+nb}{len}\PY{p}{(}\PY{n}{train\PYZus{}y}\PY{p}{)}\PY{p}{)} \PY{o}{*} \PY{p}{(}\PY{n}{np}\PY{o}{.}\PY{n}{e} \PY{o}{/} \PY{n+nb}{len}\PY{p}{(}\PY{n}{train\PYZus{}y}\PY{p}{)}\PY{p}{)}\PY{p}{)}
    \PY{n}{weights} \PY{o}{=} \PY{n}{torch}\PY{o}{.}\PY{n}{tensor}\PY{p}{(}\PY{n}{weights}\PY{p}{,} \PY{n}{dtype}\PY{o}{=}\PY{n}{torch}\PY{o}{.}\PY{n}{float32}\PY{p}{,} \PY{n}{device}\PY{o}{=}\PY{n}{device}\PY{p}{)}

\PY{k}{with} \PY{n}{torch}\PY{o}{.}\PY{n}{no\PYZus{}grad}\PY{p}{(}\PY{p}{)}\PY{p}{:}
    \PY{n}{alpha} \PY{o}{=} \PY{l+m+mi}{10}
    \PY{n}{weights2} \PY{o}{=} \PY{n}{np}\PY{o}{.}\PY{n}{tanh}\PY{p}{(} \PY{n}{alpha} \PY{o}{*}\PY{p}{(}\PY{n}{np}\PY{o}{.}\PY{n}{arange}\PY{p}{(}\PY{n+nb}{len}\PY{p}{(}\PY{n}{train\PYZus{}y}\PY{p}{)}\PY{p}{)} \PY{o}{/} \PY{n+nb}{len}\PY{p}{(}\PY{n}{train\PYZus{}y}\PY{p}{)} \PY{o}{\PYZhy{}} \PY{l+m+mi}{1}\PY{p}{)}\PY{p}{)} \PY{o}{+} \PY{l+m+mi}{1} 
    \PY{n}{weights2} \PY{o}{=} \PY{n}{torch}\PY{o}{.}\PY{n}{tensor}\PY{p}{(}\PY{n}{weights2}\PY{p}{,} \PY{n}{dtype}\PY{o}{=}\PY{n}{torch}\PY{o}{.}\PY{n}{float32}\PY{p}{,} \PY{n}{device}\PY{o}{=}\PY{n}{device}\PY{p}{)}
\end{Verbatim}
\end{tcolorbox}

    \begin{tcolorbox}[breakable, size=fbox, boxrule=1pt, pad at break*=1mm,colback=cellbackground, colframe=cellborder]
\prompt{In}{incolor}{27}{\boxspacing}
\begin{Verbatim}[commandchars=\\\{\}]
\PY{n+nb}{print}\PY{p}{(}\PY{l+s+s2}{\PYZdq{}}\PY{l+s+s2}{weights.shape: }\PY{l+s+s2}{\PYZdq{}}\PY{p}{,} \PY{n}{weights}\PY{o}{.}\PY{n}{shape}\PY{p}{)}
\end{Verbatim}
\end{tcolorbox}

    \begin{Verbatim}[commandchars=\\\{\}]
weights.shape:  torch.Size([250])
    \end{Verbatim}

    \begin{tcolorbox}[breakable, size=fbox, boxrule=1pt, pad at break*=1mm,colback=cellbackground, colframe=cellborder]
\prompt{In}{incolor}{28}{\boxspacing}
\begin{Verbatim}[commandchars=\\\{\}]
\PY{c+c1}{\PYZsh{} 画出weights}
\PY{n}{plt}\PY{o}{.}\PY{n}{figure}\PY{p}{(}\PY{n}{figsize}\PY{o}{=}\PY{p}{(}\PY{l+m+mi}{20}\PY{p}{,} \PY{l+m+mi}{10}\PY{p}{)}\PY{p}{)}
\PY{n}{plt}\PY{o}{.}\PY{n}{plot}\PY{p}{(}\PY{n}{weights}\PY{o}{.}\PY{n}{cpu}\PY{p}{(}\PY{p}{)}\PY{o}{.}\PY{n}{numpy}\PY{p}{(}\PY{p}{)}\PY{p}{)}
\PY{n}{plt}\PY{o}{.}\PY{n}{plot}\PY{p}{(}\PY{n}{weights2}\PY{o}{.}\PY{n}{cpu}\PY{p}{(}\PY{p}{)}\PY{o}{.}\PY{n}{numpy}\PY{p}{(}\PY{p}{)}\PY{p}{)}
\PY{c+c1}{\PYZsh{} 画出客流量}
\PY{n}{plt}\PY{o}{.}\PY{n}{plot}\PY{p}{(}\PY{n}{train\PYZus{}y}\PY{p}{)}
\end{Verbatim}
\end{tcolorbox}

            \begin{tcolorbox}[breakable, size=fbox, boxrule=.5pt, pad at break*=1mm, opacityfill=0]
\prompt{Out}{outcolor}{28}{\boxspacing}
\begin{Verbatim}[commandchars=\\\{\}]
[<matplotlib.lines.Line2D at 0x1d1a826f400>]
\end{Verbatim}
\end{tcolorbox}
        
    \begin{center}
    \adjustimage{max size={0.9\linewidth}{0.9\paperheight}}{output_42_1.png}
    \end{center}
    { \hspace*{\fill} \\}
    
    \hypertarget{ux5f00ux59cbux8badux7ec3}{%
\paragraph{开始训练}\label{ux5f00ux59cbux8badux7ec3}}

    \begin{tcolorbox}[breakable, size=fbox, boxrule=1pt, pad at break*=1mm,colback=cellbackground, colframe=cellborder]
\prompt{In}{incolor}{29}{\boxspacing}
\begin{Verbatim}[commandchars=\\\{\}]
\PY{n}{net}\PY{o}{.}\PY{n}{train}\PY{p}{(}\PY{p}{)}
\PY{n+nb}{print}\PY{p}{(}\PY{l+s+s2}{\PYZdq{}}\PY{l+s+s2}{Training Start}\PY{l+s+s2}{\PYZdq{}}\PY{p}{)}
\PY{k}{for} \PY{n}{epoch} \PY{o+ow}{in} \PY{n+nb}{range}\PY{p}{(}\PY{l+m+mi}{800}\PY{p}{)}\PY{p}{:}
    \PY{n}{out} \PY{o}{=} \PY{n}{net}\PY{p}{(}\PY{n}{batch\PYZus{}var\PYZus{}x}\PY{p}{)}

    \PY{c+c1}{\PYZsh{} loss = (out \PYZhy{} batch\PYZus{}var\PYZus{}y) ** 2 * weights \PYZsh{} \PYZlt{}\PYZhy{}\PYZhy{} 使用weights }
    \PY{c+c1}{\PYZsh{} test mse loss:  32856874.47789988}
    \PY{c+c1}{\PYZsh{} loss = (out \PYZhy{} batch\PYZus{}var\PYZus{}y) ** 2 * weights2 \PYZsh{} \PYZlt{}\PYZhy{}\PYZhy{} 使用weights2 }
    \PY{c+c1}{\PYZsh{} alpha = 2 test mse loss:  37580656.65421245}
    \PY{c+c1}{\PYZsh{} alpha = 5 test mse loss:  28990949.182173382}
    \PY{c+c1}{\PYZsh{} alpha = 7 test mse loss:  24158278.41746032}
    \PY{c+c1}{\PYZsh{} alpha = 10 test mse loss:  32370050.34371184}
    \PY{n}{loss} \PY{o}{=} \PY{p}{(}\PY{n}{out} \PY{o}{\PYZhy{}} \PY{n}{batch\PYZus{}var\PYZus{}y}\PY{p}{)} \PY{o}{*}\PY{o}{*} \PY{l+m+mi}{2} \PY{c+c1}{\PYZsh{} \PYZlt{}\PYZhy{}\PYZhy{} 不使用weights }
    \PY{c+c1}{\PYZsh{} test mse loss:  13283323.04859585 结果最好}

    \PY{n}{loss} \PY{o}{=} \PY{n}{loss}\PY{o}{.}\PY{n}{mean}\PY{p}{(}\PY{p}{)}
    
    \PY{n}{optimizer}\PY{o}{.}\PY{n}{zero\PYZus{}grad}\PY{p}{(}\PY{p}{)}
    \PY{n}{loss}\PY{o}{.}\PY{n}{backward}\PY{p}{(}\PY{p}{)}
    \PY{n}{optimizer}\PY{o}{.}\PY{n}{step}\PY{p}{(}\PY{p}{)}

    \PY{k}{if} \PY{n}{epoch} \PY{o}{\PYZpc{}} \PY{l+m+mi}{50} \PY{o}{==} \PY{l+m+mi}{0}\PY{p}{:}
        \PY{n+nb}{print}\PY{p}{(}\PY{l+s+s1}{\PYZsq{}}\PY{l+s+s1}{Epoch: }\PY{l+s+si}{\PYZob{}:4\PYZcb{}}\PY{l+s+s1}{, Loss: }\PY{l+s+si}{\PYZob{}:.5f\PYZcb{}}\PY{l+s+s1}{\PYZsq{}}\PY{o}{.}\PY{n}{format}\PY{p}{(}\PY{n}{epoch}\PY{p}{,} \PY{n}{loss}\PY{o}{.}\PY{n}{item}\PY{p}{(}\PY{p}{)}\PY{p}{)}\PY{p}{)}
\PY{n+nb}{print}\PY{p}{(}\PY{l+s+s2}{\PYZdq{}}\PY{l+s+s2}{Training End, Loss: }\PY{l+s+si}{\PYZob{}:.5f\PYZcb{}}\PY{l+s+s2}{\PYZdq{}}\PY{o}{.}\PY{n}{format}\PY{p}{(}\PY{n}{loss}\PY{o}{.}\PY{n}{item}\PY{p}{(}\PY{p}{)}\PY{p}{)}\PY{p}{)}
\end{Verbatim}
\end{tcolorbox}

    \begin{Verbatim}[commandchars=\\\{\}]
Training Start
Epoch:    0, Loss: 0.37251
Epoch:   50, Loss: 0.01125
Epoch:  100, Loss: 0.00961
Epoch:  150, Loss: 0.00889
Epoch:  200, Loss: 0.00834
Epoch:  250, Loss: 0.00785
Epoch:  300, Loss: 0.00748
Epoch:  350, Loss: 0.00717
Epoch:  400, Loss: 0.00699
Epoch:  450, Loss: 0.00691
Epoch:  500, Loss: 0.00680
Epoch:  550, Loss: 0.00672
Epoch:  600, Loss: 0.00664
Epoch:  650, Loss: 0.00657
Epoch:  700, Loss: 0.00685
Epoch:  750, Loss: 0.00643
Training End, Loss: 0.00636
    \end{Verbatim}

    保存参数

    \begin{tcolorbox}[breakable, size=fbox, boxrule=1pt, pad at break*=1mm,colback=cellbackground, colframe=cellborder]
\prompt{In}{incolor}{30}{\boxspacing}
\begin{Verbatim}[commandchars=\\\{\}]
\PY{c+c1}{\PYZsh{} torch.save(net.state\PYZus{}dict(), \PYZsq{}\PYZob{}\PYZcb{}/net.pth\PYZsq{}.format(mod\PYZus{}dir))}
\PY{c+c1}{\PYZsh{} print(\PYZdq{}Save in:\PYZdq{}, \PYZsq{}\PYZob{}\PYZcb{}/net.pth\PYZsq{}.format(mod\PYZus{}dir))}
\end{Verbatim}
\end{tcolorbox}

    \hypertarget{eval}{%
\subsubsection{eval}\label{eval}}

    制作test set

    \begin{tcolorbox}[breakable, size=fbox, boxrule=1pt, pad at break*=1mm,colback=cellbackground, colframe=cellborder]
\prompt{In}{incolor}{31}{\boxspacing}
\begin{Verbatim}[commandchars=\\\{\}]
\PY{n}{test\PYZus{}x} \PY{o}{=} \PY{n}{seq\PYZus{}norm}\PY{o}{.}\PY{n}{copy}\PY{p}{(}\PY{p}{)} 
\PY{n}{test\PYZus{}x}\PY{p}{[}\PY{n}{train\PYZus{}size}\PY{p}{:}\PY{p}{,} \PY{l+m+mi}{0}\PY{p}{]} \PY{o}{=} \PY{l+m+mi}{0} \PY{c+c1}{\PYZsh{} 设置为0,表示预测 }
\PY{n}{test\PYZus{}x} \PY{o}{=} \PY{n}{test\PYZus{}x}\PY{p}{[}\PY{p}{:}\PY{p}{,} \PY{n}{np}\PY{o}{.}\PY{n}{newaxis}\PY{p}{,} \PY{p}{:}\PY{p}{]} 
\PY{n}{test\PYZus{}x} \PY{o}{=} \PY{n}{torch}\PY{o}{.}\PY{n}{tensor}\PY{p}{(}\PY{n}{test\PYZus{}x}\PY{p}{,} \PY{n}{dtype}\PY{o}{=}\PY{n}{torch}\PY{o}{.}\PY{n}{float32}\PY{p}{,} \PY{n}{device}\PY{o}{=}\PY{n}{device}\PY{p}{)}
\PY{n+nb}{print}\PY{p}{(}\PY{l+s+s2}{\PYZdq{}}\PY{l+s+s2}{test\PYZus{}x.shape: }\PY{l+s+s2}{\PYZdq{}}\PY{p}{,} \PY{n}{test\PYZus{}x}\PY{o}{.}\PY{n}{shape}\PY{p}{)}
\end{Verbatim}
\end{tcolorbox}

    \begin{Verbatim}[commandchars=\\\{\}]
test\_x.shape:  torch.Size([359, 1, 1])
    \end{Verbatim}

    \begin{tcolorbox}[breakable, size=fbox, boxrule=1pt, pad at break*=1mm,colback=cellbackground, colframe=cellborder]
\prompt{In}{incolor}{32}{\boxspacing}
\begin{Verbatim}[commandchars=\\\{\}]
\PY{c+c1}{\PYZsh{} net.load\PYZus{}state\PYZus{}dict(torch.load(\PYZsq{}\PYZob{}\PYZcb{}/net.pth\PYZsq{}.format(mod\PYZus{}dir), map\PYZus{}location=lambda storage, loc: storage))}
\PY{n}{net}\PY{o}{.}\PY{n}{eval}\PY{p}{(}\PY{p}{)}
\end{Verbatim}
\end{tcolorbox}

            \begin{tcolorbox}[breakable, size=fbox, boxrule=.5pt, pad at break*=1mm, opacityfill=0]
\prompt{Out}{outcolor}{32}{\boxspacing}
\begin{Verbatim}[commandchars=\\\{\}]
RegLSTM(
  (rnn): LSTM(1, 10)
  (reg): Sequential(
    (0): Linear(in\_features=10, out\_features=10, bias=True)
    (1): Tanh()
    (2): Linear(in\_features=10, out\_features=1, bias=True)
  )
)
\end{Verbatim}
\end{tcolorbox}
        
    \begin{tcolorbox}[breakable, size=fbox, boxrule=1pt, pad at break*=1mm,colback=cellbackground, colframe=cellborder]
\prompt{In}{incolor}{33}{\boxspacing}
\begin{Verbatim}[commandchars=\\\{\}]
\PY{k}{with} \PY{n}{torch}\PY{o}{.}\PY{n}{no\PYZus{}grad}\PY{p}{(}\PY{p}{)}\PY{p}{:}
    \PY{l+s+sd}{\PYZsq{}\PYZsq{}\PYZsq{}simple way is elegant\PYZsq{}\PYZsq{}\PYZsq{}}
    \PY{k}{for} \PY{n}{i} \PY{o+ow}{in} \PY{n+nb}{range}\PY{p}{(}\PY{n}{train\PYZus{}size}\PY{p}{,} \PY{n+nb}{len}\PY{p}{(}\PY{n}{data}\PY{p}{)}\PY{p}{)}\PY{p}{:}
        \PY{n}{test\PYZus{}y} \PY{o}{=} \PY{n}{net}\PY{p}{(}\PY{n}{test\PYZus{}x}\PY{p}{[}\PY{p}{:}\PY{n}{i}\PY{p}{]}\PY{p}{)}
        \PY{n}{test\PYZus{}x}\PY{p}{[}\PY{n}{i}\PY{p}{,} \PY{l+m+mi}{0}\PY{p}{,} \PY{l+m+mi}{0}\PY{p}{]} \PY{o}{=} \PY{n}{test\PYZus{}y}\PY{p}{[}\PY{o}{\PYZhy{}}\PY{l+m+mi}{1}\PY{p}{]}

    \PY{n}{pred\PYZus{}y} \PY{o}{=} \PY{n}{test\PYZus{}x}\PY{p}{[}\PY{l+m+mi}{1}\PY{p}{:}\PY{p}{,} \PY{l+m+mi}{0}\PY{p}{,} \PY{l+m+mi}{0}\PY{p}{]}
    \PY{n}{pred\PYZus{}y} \PY{o}{=} \PY{n}{pred\PYZus{}y}\PY{o}{.}\PY{n}{cpu}\PY{p}{(}\PY{p}{)}\PY{o}{.}\PY{n}{data}\PY{o}{.}\PY{n}{numpy}\PY{p}{(}\PY{p}{)}

    \PY{n}{diff\PYZus{}y} \PY{o}{=} \PY{n}{pred\PYZus{}y}\PY{p}{[}\PY{n}{train\PYZus{}size}\PY{p}{:}\PY{p}{]} \PY{o}{\PYZhy{}} \PY{n}{data\PYZus{}y}\PY{p}{[}\PY{n}{train\PYZus{}size}\PY{p}{:}\PY{p}{]}
    \PY{n}{l2\PYZus{}loss} \PY{o}{=} \PY{n}{np}\PY{o}{.}\PY{n}{mean}\PY{p}{(}\PY{n}{diff\PYZus{}y} \PY{o}{*}\PY{o}{*} \PY{l+m+mi}{2}\PY{p}{)}
    \PY{n+nb}{print}\PY{p}{(}\PY{l+s+s2}{\PYZdq{}}\PY{l+s+s2}{ test mse loss: }\PY{l+s+s2}{\PYZdq{}}\PY{p}{,} \PY{n}{l2\PYZus{}loss}\PY{p}{)}
\end{Verbatim}
\end{tcolorbox}

    \begin{Verbatim}[commandchars=\\\{\}]
 test mse loss:  0.007266506
    \end{Verbatim}

    \hypertarget{for-data-1}{%
\subparagraph{for data 1}\label{for-data-1}}

    \begin{tcolorbox}[breakable, size=fbox, boxrule=1pt, pad at break*=1mm,colback=cellbackground, colframe=cellborder]
\prompt{In}{incolor}{34}{\boxspacing}
\begin{Verbatim}[commandchars=\\\{\}]
\PY{c+c1}{\PYZsh{} origin\PYZus{}data = df.values}
\PY{c+c1}{\PYZsh{} \PYZsh{} 反转}
\PY{c+c1}{\PYZsh{} origin\PYZus{}data = origin\PYZus{}data[::\PYZhy{}1]}
\end{Verbatim}
\end{tcolorbox}

    \hypertarget{for-data-2}{%
\subparagraph{for data 2}\label{for-data-2}}

    \begin{tcolorbox}[breakable, size=fbox, boxrule=1pt, pad at break*=1mm,colback=cellbackground, colframe=cellborder]
\prompt{In}{incolor}{35}{\boxspacing}
\begin{Verbatim}[commandchars=\\\{\}]
\PY{n}{origin\PYZus{}data} \PY{o}{=} \PY{n}{seq}\PY{o}{.}\PY{n}{copy}\PY{p}{(}\PY{p}{)}
\end{Verbatim}
\end{tcolorbox}

    \begin{tcolorbox}[breakable, size=fbox, boxrule=1pt, pad at break*=1mm,colback=cellbackground, colframe=cellborder]
\prompt{In}{incolor}{36}{\boxspacing}
\begin{Verbatim}[commandchars=\\\{\}]
\PY{n+nb}{print}\PY{p}{(}\PY{l+s+s2}{\PYZdq{}}\PY{l+s+s2}{pred\PYZus{}y.shape: }\PY{l+s+s2}{\PYZdq{}}\PY{p}{,} \PY{n}{pred\PYZus{}y}\PY{o}{.}\PY{n}{shape}\PY{p}{)}
\PY{n+nb}{print}\PY{p}{(}\PY{l+s+s2}{\PYZdq{}}\PY{l+s+s2}{data\PYZus{}y.shape: }\PY{l+s+s2}{\PYZdq{}}\PY{p}{,} \PY{n}{data\PYZus{}y}\PY{o}{.}\PY{n}{shape}\PY{p}{)}
\PY{n+nb}{print}\PY{p}{(}\PY{l+s+s2}{\PYZdq{}}\PY{l+s+s2}{the last pred\PYZus{}y: }\PY{l+s+s2}{\PYZdq{}}\PY{p}{,} \PY{n}{pred\PYZus{}y}\PY{p}{[}\PY{o}{\PYZhy{}}\PY{l+m+mi}{1}\PY{p}{]}\PY{p}{)}
\PY{n+nb}{print}\PY{p}{(}\PY{l+s+s2}{\PYZdq{}}\PY{l+s+s2}{the last data\PYZus{}y: }\PY{l+s+s2}{\PYZdq{}}\PY{p}{,} \PY{n}{data\PYZus{}y}\PY{p}{[}\PY{o}{\PYZhy{}}\PY{l+m+mi}{1}\PY{p}{]}\PY{p}{)}
\end{Verbatim}
\end{tcolorbox}

    \begin{Verbatim}[commandchars=\\\{\}]
pred\_y.shape:  (358,)
data\_y.shape:  (358,)
the last pred\_y:  -0.646692
the last data\_y:  -0.79226667
    \end{Verbatim}

    \hypertarget{ux6570ux636eux8fd8ux539f}{%
\subparagraph{数据还原}\label{ux6570ux636eux8fd8ux539f}}

    \begin{tcolorbox}[breakable, size=fbox, boxrule=1pt, pad at break*=1mm,colback=cellbackground, colframe=cellborder]
\prompt{In}{incolor}{37}{\boxspacing}
\begin{Verbatim}[commandchars=\\\{\}]
\PY{n}{mean} \PY{o}{=} \PY{n}{origin\PYZus{}data}\PY{o}{.}\PY{n}{mean}\PY{p}{(}\PY{p}{)}
\PY{n}{std} \PY{o}{=} \PY{n}{origin\PYZus{}data}\PY{o}{.}\PY{n}{std}\PY{p}{(}\PY{p}{)}
\PY{n+nb}{print}\PY{p}{(}\PY{l+s+s2}{\PYZdq{}}\PY{l+s+s2}{mean: }\PY{l+s+s2}{\PYZdq{}}\PY{p}{,} \PY{n}{mean}\PY{p}{)}
\PY{n+nb}{print}\PY{p}{(}\PY{l+s+s2}{\PYZdq{}}\PY{l+s+s2}{std: }\PY{l+s+s2}{\PYZdq{}}\PY{p}{,} \PY{n}{std}\PY{p}{)}

\PY{c+c1}{\PYZsh{} 将预测值还原}
\PY{n}{pred\PYZus{}y} \PY{o}{=} \PY{n}{pred\PYZus{}y} \PY{o}{*} \PY{n}{std} \PY{o}{+} \PY{n}{mean}
\PY{c+c1}{\PYZsh{} 设置成int类型}
\PY{n}{pred\PYZus{}y} \PY{o}{=} \PY{n}{pred\PYZus{}y}\PY{o}{.}\PY{n}{astype}\PY{p}{(}\PY{n+nb}{int}\PY{p}{)}
\end{Verbatim}
\end{tcolorbox}

    \begin{Verbatim}[commandchars=\\\{\}]
mean:  90972.805
std:  89102.33
    \end{Verbatim}

    \hypertarget{ux6c42mse-mean-squared-error}{%
\subparagraph{求MSE (Mean Squared
Error)}\label{ux6c42mse-mean-squared-error}}

    \begin{tcolorbox}[breakable, size=fbox, boxrule=1pt, pad at break*=1mm,colback=cellbackground, colframe=cellborder]
\prompt{In}{incolor}{38}{\boxspacing}
\begin{Verbatim}[commandchars=\\\{\}]
\PY{c+c1}{\PYZsh{} 和真实值对比求误差}
\PY{n}{diff\PYZus{}y} \PY{o}{=} \PY{n}{pred\PYZus{}y}\PY{p}{[}\PY{n}{train\PYZus{}size}\PY{p}{:}\PY{p}{]} \PY{o}{\PYZhy{}} \PY{n}{origin\PYZus{}data}\PY{p}{[}\PY{n}{train\PYZus{}size}\PY{p}{:}\PY{p}{]}
\PY{n}{l2\PYZus{}loss} \PY{o}{=} \PY{n}{np}\PY{o}{.}\PY{n}{mean}\PY{p}{(}\PY{n}{diff\PYZus{}y} \PY{o}{*}\PY{o}{*} \PY{l+m+mi}{2}\PY{p}{)}
\PY{n+nb}{print}\PY{p}{(}\PY{l+s+s2}{\PYZdq{}}\PY{l+s+s2}{ test mse loss: }\PY{l+s+s2}{\PYZdq{}}\PY{p}{,} \PY{n}{l2\PYZus{}loss}\PY{p}{)}
\end{Verbatim}
\end{tcolorbox}

    \begin{Verbatim}[commandchars=\\\{\}]
 test mse loss:  53422339.48335032
    \end{Verbatim}

    \hypertarget{ux9884ux6d4bux66f2ux7ebfux56fe}{%
\subparagraph{预测曲线图}\label{ux9884ux6d4bux66f2ux7ebfux56fe}}

    \begin{tcolorbox}[breakable, size=fbox, boxrule=1pt, pad at break*=1mm,colback=cellbackground, colframe=cellborder]
\prompt{In}{incolor}{39}{\boxspacing}
\begin{Verbatim}[commandchars=\\\{\}]
\PY{n}{plt}\PY{o}{.}\PY{n}{figure}\PY{p}{(}\PY{n}{figsize}\PY{o}{=}\PY{p}{(}\PY{l+m+mi}{20}\PY{p}{,}\PY{l+m+mi}{10}\PY{p}{)}\PY{p}{)}
\PY{n}{plt}\PY{o}{.}\PY{n}{plot}\PY{p}{(}\PY{n}{pred\PYZus{}y}\PY{p}{,} \PY{l+s+s1}{\PYZsq{}}\PY{l+s+s1}{r}\PY{l+s+s1}{\PYZsq{}}\PY{p}{,} \PY{n}{label}\PY{o}{=}\PY{l+s+s1}{\PYZsq{}}\PY{l+s+s1}{pred}\PY{l+s+s1}{\PYZsq{}}\PY{p}{)}
\PY{n}{plt}\PY{o}{.}\PY{n}{plot}\PY{p}{(}\PY{n}{origin\PYZus{}data}\PY{p}{,} \PY{l+s+s1}{\PYZsq{}}\PY{l+s+s1}{b}\PY{l+s+s1}{\PYZsq{}}\PY{p}{,} \PY{n}{label}\PY{o}{=}\PY{l+s+s1}{\PYZsq{}}\PY{l+s+s1}{real}\PY{l+s+s1}{\PYZsq{}}\PY{p}{,} \PY{n}{alpha}\PY{o}{=}\PY{l+m+mf}{0.3}\PY{p}{)}
\PY{n}{plt}\PY{o}{.}\PY{n}{plot}\PY{p}{(}\PY{p}{[}\PY{n}{train\PYZus{}size}\PY{p}{,} \PY{n}{train\PYZus{}size}\PY{p}{]}\PY{p}{,} \PY{p}{[}\PY{l+m+mi}{0}\PY{p}{,} \PY{l+m+mi}{350000}\PY{p}{]}\PY{p}{,} \PY{n}{color}\PY{o}{=}\PY{l+s+s1}{\PYZsq{}}\PY{l+s+s1}{k}\PY{l+s+s1}{\PYZsq{}}\PY{p}{,} \PY{n}{label}\PY{o}{=}\PY{l+s+s1}{\PYZsq{}}\PY{l+s+s1}{train | pred}\PY{l+s+s1}{\PYZsq{}}\PY{p}{,} \PY{n}{linestyle}\PY{o}{=}\PY{l+s+s1}{\PYZsq{}}\PY{l+s+s1}{\PYZhy{}\PYZhy{}}\PY{l+s+s1}{\PYZsq{}}\PY{p}{)}
\PY{n}{plt}\PY{o}{.}\PY{n}{legend}\PY{p}{(}\PY{n}{loc}\PY{o}{=}\PY{l+s+s1}{\PYZsq{}}\PY{l+s+s1}{best}\PY{l+s+s1}{\PYZsq{}}\PY{p}{)}
\PY{c+c1}{\PYZsh{} plt.savefig(\PYZsq{}pred\PYZus{}with\PYZus{}regularization\PYZus{}batch\PYZus{}lstm1.png\PYZsq{})}
\end{Verbatim}
\end{tcolorbox}

            \begin{tcolorbox}[breakable, size=fbox, boxrule=.5pt, pad at break*=1mm, opacityfill=0]
\prompt{Out}{outcolor}{39}{\boxspacing}
\begin{Verbatim}[commandchars=\\\{\}]
<matplotlib.legend.Legend at 0x1d1a82f2250>
\end{Verbatim}
\end{tcolorbox}
        
    \begin{center}
    \adjustimage{max size={0.9\linewidth}{0.9\paperheight}}{output_62_1.png}
    \end{center}
    { \hspace*{\fill} \\}
    
    \hypertarget{ux957fux671fux9884ux6d4b}{%
\paragraph{长期预测}\label{ux957fux671fux9884ux6d4b}}

    \begin{tcolorbox}[breakable, size=fbox, boxrule=1pt, pad at break*=1mm,colback=cellbackground, colframe=cellborder]
\prompt{In}{incolor}{40}{\boxspacing}
\begin{Verbatim}[commandchars=\\\{\}]
\PY{n}{pred\PYZus{}result} \PY{o}{=} \PY{n}{seq\PYZus{}norm}\PY{o}{.}\PY{n}{copy}\PY{p}{(}\PY{p}{)}
\PY{n}{pred\PYZus{}result} \PY{o}{=} \PY{n}{np}\PY{o}{.}\PY{n}{concatenate}\PY{p}{(}\PY{p}{(}\PY{n}{pred\PYZus{}result}\PY{p}{,} \PY{n}{np}\PY{o}{.}\PY{n}{zeros}\PY{p}{(}\PY{p}{(}\PY{l+m+mi}{60}\PY{p}{,} \PY{l+m+mi}{1}\PY{p}{)}\PY{p}{)}\PY{p}{)}\PY{p}{,} \PY{n}{axis}\PY{o}{=}\PY{l+m+mi}{0}\PY{p}{)}
\PY{n}{pred\PYZus{}result}\PY{p}{[}\PY{n}{train\PYZus{}size}\PY{p}{:}\PY{p}{,} \PY{l+m+mi}{0}\PY{p}{]} \PY{o}{=} \PY{l+m+mi}{0} \PY{c+c1}{\PYZsh{} 设置为0,表示预测 }
\PY{n}{pred\PYZus{}result} \PY{o}{=} \PY{n}{pred\PYZus{}result}\PY{p}{[}\PY{p}{:}\PY{p}{,} \PY{n}{np}\PY{o}{.}\PY{n}{newaxis}\PY{p}{,} \PY{p}{:}\PY{p}{]} 
\PY{n}{pred\PYZus{}result} \PY{o}{=} \PY{n}{torch}\PY{o}{.}\PY{n}{tensor}\PY{p}{(}\PY{n}{pred\PYZus{}result}\PY{p}{,} \PY{n}{dtype}\PY{o}{=}\PY{n}{torch}\PY{o}{.}\PY{n}{float32}\PY{p}{,} \PY{n}{device}\PY{o}{=}\PY{n}{device}\PY{p}{)}

\PY{k}{with} \PY{n}{torch}\PY{o}{.}\PY{n}{no\PYZus{}grad}\PY{p}{(}\PY{p}{)}\PY{p}{:}
    \PY{l+s+sd}{\PYZsq{}\PYZsq{}\PYZsq{}simple way is elegant\PYZsq{}\PYZsq{}\PYZsq{}}
    \PY{k}{for} \PY{n}{i} \PY{o+ow}{in} \PY{n+nb}{range}\PY{p}{(}\PY{n}{train\PYZus{}size}\PY{p}{,} \PY{n+nb}{len}\PY{p}{(}\PY{n}{data}\PY{p}{)}\PY{o}{+}\PY{l+m+mi}{60}\PY{p}{)}\PY{p}{:}
        \PY{n}{test\PYZus{}y} \PY{o}{=} \PY{n}{net}\PY{p}{(}\PY{n}{pred\PYZus{}result}\PY{p}{[}\PY{p}{:}\PY{n}{i}\PY{p}{]}\PY{p}{)}
        \PY{n}{pred\PYZus{}result}\PY{p}{[}\PY{n}{i}\PY{p}{,} \PY{l+m+mi}{0}\PY{p}{,} \PY{l+m+mi}{0}\PY{p}{]} \PY{o}{=} \PY{n}{test\PYZus{}y}\PY{p}{[}\PY{o}{\PYZhy{}}\PY{l+m+mi}{1}\PY{p}{]}

    \PY{n}{pred\PYZus{}y} \PY{o}{=} \PY{n}{pred\PYZus{}result}\PY{p}{[}\PY{l+m+mi}{1}\PY{p}{:}\PY{p}{,} \PY{l+m+mi}{0}\PY{p}{,} \PY{l+m+mi}{0}\PY{p}{]}
    \PY{n}{pred\PYZus{}y} \PY{o}{=} \PY{n}{pred\PYZus{}y}\PY{o}{.}\PY{n}{cpu}\PY{p}{(}\PY{p}{)}\PY{o}{.}\PY{n}{data}\PY{o}{.}\PY{n}{numpy}\PY{p}{(}\PY{p}{)}
\end{Verbatim}
\end{tcolorbox}

    \begin{tcolorbox}[breakable, size=fbox, boxrule=1pt, pad at break*=1mm,colback=cellbackground, colframe=cellborder]
\prompt{In}{incolor}{41}{\boxspacing}
\begin{Verbatim}[commandchars=\\\{\}]
\PY{c+c1}{\PYZsh{} 还原}
\PY{n}{pred\PYZus{}y} \PY{o}{=} \PY{n}{pred\PYZus{}y} \PY{o}{*} \PY{n}{std} \PY{o}{+} \PY{n}{mean}
\PY{n}{pred\PYZus{}y} \PY{o}{=} \PY{n}{pred\PYZus{}y}\PY{o}{.}\PY{n}{astype}\PY{p}{(}\PY{n+nb}{int}\PY{p}{)}
\end{Verbatim}
\end{tcolorbox}

    \begin{tcolorbox}[breakable, size=fbox, boxrule=1pt, pad at break*=1mm,colback=cellbackground, colframe=cellborder]
\prompt{In}{incolor}{42}{\boxspacing}
\begin{Verbatim}[commandchars=\\\{\}]
\PY{n}{plt}\PY{o}{.}\PY{n}{figure}\PY{p}{(}\PY{n}{figsize}\PY{o}{=}\PY{p}{(}\PY{l+m+mi}{20}\PY{p}{,}\PY{l+m+mi}{10}\PY{p}{)}\PY{p}{)}
\PY{n}{plt}\PY{o}{.}\PY{n}{plot}\PY{p}{(}\PY{n}{pred\PYZus{}y}\PY{p}{,} \PY{l+s+s1}{\PYZsq{}}\PY{l+s+s1}{r}\PY{l+s+s1}{\PYZsq{}}\PY{p}{,} \PY{n}{label}\PY{o}{=}\PY{l+s+s1}{\PYZsq{}}\PY{l+s+s1}{pred}\PY{l+s+s1}{\PYZsq{}}\PY{p}{)}
\PY{n}{plt}\PY{o}{.}\PY{n}{plot}\PY{p}{(}\PY{n}{origin\PYZus{}data}\PY{p}{,} \PY{l+s+s1}{\PYZsq{}}\PY{l+s+s1}{b}\PY{l+s+s1}{\PYZsq{}}\PY{p}{,} \PY{n}{label}\PY{o}{=}\PY{l+s+s1}{\PYZsq{}}\PY{l+s+s1}{real}\PY{l+s+s1}{\PYZsq{}}\PY{p}{,} \PY{n}{alpha}\PY{o}{=}\PY{l+m+mf}{0.3}\PY{p}{)}
\PY{n}{plt}\PY{o}{.}\PY{n}{plot}\PY{p}{(}\PY{p}{[}\PY{n}{train\PYZus{}size}\PY{p}{,} \PY{n}{train\PYZus{}size}\PY{p}{]}\PY{p}{,} \PY{p}{[}\PY{l+m+mi}{0}\PY{p}{,} \PY{l+m+mi}{350000}\PY{p}{]}\PY{p}{,} \PY{n}{color}\PY{o}{=}\PY{l+s+s1}{\PYZsq{}}\PY{l+s+s1}{k}\PY{l+s+s1}{\PYZsq{}}\PY{p}{,} \PY{n}{label}\PY{o}{=}\PY{l+s+s1}{\PYZsq{}}\PY{l+s+s1}{train | pred}\PY{l+s+s1}{\PYZsq{}}\PY{p}{,} \PY{n}{linestyle}\PY{o}{=}\PY{l+s+s1}{\PYZsq{}}\PY{l+s+s1}{\PYZhy{}\PYZhy{}}\PY{l+s+s1}{\PYZsq{}}\PY{p}{)}
\PY{n}{plt}\PY{o}{.}\PY{n}{legend}\PY{p}{(}\PY{n}{loc}\PY{o}{=}\PY{l+s+s1}{\PYZsq{}}\PY{l+s+s1}{best}\PY{l+s+s1}{\PYZsq{}}\PY{p}{)}
\PY{c+c1}{\PYZsh{} plt.savefig(\PYZsq{}pred\PYZus{}with\PYZus{}weight2\PYZus{}alpha\PYZus{}\PYZsq{} + str(alpha) + \PYZsq{}.png\PYZsq{})}
\PY{c+c1}{\PYZsh{} plt.savefig(\PYZsq{}pred\PYZus{}with\PYZus{}no\PYZus{}weight\PYZus{}batchsize\PYZus{}\PYZsq{} + str(batch\PYZus{}size) + \PYZsq{}.png\PYZsq{})}
\end{Verbatim}
\end{tcolorbox}

            \begin{tcolorbox}[breakable, size=fbox, boxrule=.5pt, pad at break*=1mm, opacityfill=0]
\prompt{Out}{outcolor}{42}{\boxspacing}
\begin{Verbatim}[commandchars=\\\{\}]
<matplotlib.legend.Legend at 0x1d1a9383940>
\end{Verbatim}
\end{tcolorbox}
        
    \begin{center}
    \adjustimage{max size={0.9\linewidth}{0.9\paperheight}}{output_66_1.png}
    \end{center}
    { \hspace*{\fill} \\}
    
    \begin{tcolorbox}[breakable, size=fbox, boxrule=1pt, pad at break*=1mm,colback=cellbackground, colframe=cellborder]
\prompt{In}{incolor}{43}{\boxspacing}
\begin{Verbatim}[commandchars=\\\{\}]
\PY{n+nb}{print}\PY{p}{(}\PY{l+s+s2}{\PYZdq{}}\PY{l+s+s2}{pred\PYZus{}y.shape: }\PY{l+s+s2}{\PYZdq{}}\PY{p}{,} \PY{n}{pred\PYZus{}y}\PY{o}{.}\PY{n}{shape}\PY{p}{)}
\PY{n+nb}{print}\PY{p}{(}\PY{l+s+s2}{\PYZdq{}}\PY{l+s+s2}{final pred\PYZus{}y: }\PY{l+s+s2}{\PYZdq{}}\PY{p}{,} \PY{n}{pred\PYZus{}y}\PY{p}{[}\PY{o}{\PYZhy{}}\PY{l+m+mi}{1}\PY{p}{]}\PY{p}{)}
\end{Verbatim}
\end{tcolorbox}

    \begin{Verbatim}[commandchars=\\\{\}]
pred\_y.shape:  (418,)
final pred\_y:  33352
    \end{Verbatim}

    \hypertarget{ux603bux8ba1ux6539ux8fdb}{%
\subsubsection{总计改进}\label{ux603bux8ba1ux6539ux8fdb}}

    由上面可以发现不加weights训练效果是最好的,我们多次进行不加weights的训练查看效果。

我们下一步可以做的工作有, -
制作mini\_batch,训练更多的数据而不是只有我们训练的一个batch而已 -
重新确定好训练集和测试集的比例分割数据


最优的一幅图

\begin{center}
	\adjustimage{max size={0.9\linewidth}{0.9\paperheight}}{pred_with_regularization_batch_lstm2.png}
\end{center}



    % Add a bibliography block to the postdoc
    
    
    
\end{document}
